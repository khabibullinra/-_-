% хороший раздел - но тут сложно придумать хорошие задачи 
% на практике дебитам либо верим и используем, либо не верим
% дебитограммы никто нынче особо не смотрит 
% хотя можно сделать блок по построению дебитограмм из синтетических данных
% и примеры с интерпретацией соответственно 
\section{Теория}
\begin{itemize}
    \item Принципы измерения дебита (расхода). Оборудования для измерения дебита на поверхности и в скважине
    \item Замер обводненности и газового фактора продукции
    \item Замеры количества механических примесей
    \item Профили притока к скважине. Замер профиля притока с использованием скважинного оборудования
\end{itemize}

\section{Задачи}

\begin{itemize}
    \item Оценка дебита по закону Бернулли - на гидравлическом сужении. Оценка дебита по характеристике потока через штуцер
    \item Оценка обводненности по данным замером плотности флюидов и плотности смеси
    \item Оценка газового фактора по данным эксплуатации скважины - перепад давления в скважине и на штуцере
    \item Оценка газового фактора по данным отжима динамического уровня
    \item Расчет профиля изменения давления и профиля притока в горизонтальной скважине (учет по плотности притока и диаметру ствола скважины)
    
\end{itemize}
%\section{Выявление нарушений герметичности ствола скважины}
    