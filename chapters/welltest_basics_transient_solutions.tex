Для установившегося режима фильтрации давление в пласте не меняется. Для псевдо-установившегося режима постоянным остается перепад давления между пластом и забоем. После запуска, остановки или изменения режима работы скважины эти условия не выполняются. Давление в различных точках пласта может меняться по разному. Такой режим называют неустановившимся, а решения его описывающие нестационарными (зависят от времени).

\subsubsection{Безразмерные переменные}

Часто для анализа уравнений неустановившейся фильтрации используются безразмерные переменные 

\marginpar{
	\href{https://en.wikipedia.org/wiki/Nondimensionalization}{Обезразмеривание на en.wikipedia.org} 
	\includegraphics[scale=0.4]{pics/qr_Nondimensionalization.eps} 
}

$$ r_D = \frac{r}{r_w} $$
$$ t_D = \frac{kt}{\phi \mu c_t r_w^2}$$
$$ p_D = \frac{2 \pi kh}{q_s B \mu} \left( p_i - p_{wf} \right) $$

Здесь использование единицы измерения СИ. 

$q_s$ - дебит скважины на поверхности, приведенный к нормальным условиям м3/с

$\phi$ - пористость, доли единиц

$\mu$ - вязкость нефти в пласте, Па с

$B$ - объемный коэффициент нефти, м3/м3

$p_i$ - начальное давление в пласте, Па

$p_{wf}$ - давление забойное, Па

$c_t$ - общая сжимаемость системы в пласте, 1/Па

Использование безразмерных переменных позволяет упростить уравнение фильтрации, которое примет вид

$$ \frac{\partial p_D}{ \partial t_D} = \frac{1}{r_D} \frac{ \partial{ \left( r_D \dfrac{\partial p_D}{ \partial r_D} \right) } }{ \partial{r_D} } $$

Решение этого уравнения - функция безразмерного давления от безразмерных времени и расстояния $p_D(r_D, t_D) $

Для практических расчетов удобнее бывает использовать безразмерные переменные полученные для практических метрических единиц измерения. 
$$ r_D = \frac{r}{r_w} $$
$$ t_D = \frac{0.00036 kt}{\phi \mu c_t r_w^2}$$
$$ p_D = \frac{kh}{ 18.41 q_s B \mu} \left( p_i - p_{wf} \right) $$

Здесь использование единицы измерения СИ. 

$q_s$ - дебит скважины на поверхности, приведенный к нормальным условиям м3/сут

$\phi$ - пористость, доли единиц

$\mu$ - вязкость нефти в пласте, сП

$B$ - объемный коэффициент нефти, м3/м3

$p_i$ - начальное давление в пласте, атм

$p_{wf}$ - давление забойное, атм

$c_t$ - общая сжимаемость системы в пласте, 1/атм

дополнительно можно ввести безразмерный коэффициент влияния ствола скважины
$$ C_D = \frac{0.159}{ h \phi \mu c_t r_w^2 } C_s $$

\subsubsection{Решение линейного стока}

Для решения уравнения фильтрации - линейного дифференциального уравнения в частных производных второго порядка необходимо задать начальные и граничные условия. 

Самое простое решение можно получить для случая вертикальной скважины бесконечно малого радиуса запускающейся с постоянным дебитом. Условия соответствующие этому случаю можно выразить следующим образом:

\begin{itemize}
	\item Начальное условие. До запуска скважины в момент времени  $t_D = 0$ давление в пласте равно начальному во всех точках $p=p_i$
	$$ t_D < 0, p_D = 0 $$ 
	\item Граничное условие на скважине.  Условие постоянства дебита на скважине можно трансформировать в граничное условие опираясь на закон Дарси.
	$$ \lim_{r_D \to 0} {r_D \frac{\partial p_D}{\partial r_D}} = -1$$
	\item Граничное условие на бесконечном расстоянии от скважины. Давление в пласте на бесконечно большом расстоянии от скважины равно начальному.
	$$ r_D = \infty, p_D = 0$$
\end{itemize}

\begin{wrapfigure}{r}{0.5\textwidth}
	%\begin{figure}[h!]
	\begin{center}
		\begin{tikzpicture}
			\begin{axis}
				[axis lines = left,
				 width = 0.48\textwidth,
				%xlabel=$x$,
				%ylabel={$Ei_1(x)$},
				]
				\addplot gnuplot[no markers, samples=100, domain = 0:10]{expint(1,x)};
				\addlegendentry{$Ei_1(x)$}
			\end{axis}
		\end{tikzpicture}
		\caption{График функции интегральной экспоненты $Ei_1(x)$.}
		\label{ris:ei1}
	\end{center}
	%\end{figure}
\end{wrapfigure}



В этом случае решение может быть выражено через функцию интегральной экспоненты 
$$ p_D(r_D,t_D) = - \frac{1}{2} Ei \left(- \dfrac{ r_D^2}{4t_d} \right)$$

где $-Ei(-x)$ - интегральная показательная функция.

$$Ei(x)=-\int\limits_{x}^{\infty}\frac{e^{-t}}{t}\,\mathrm dt$$

\marginpar{
	\href{https://www.wolframalpha.com/input/?i=Ei\%28x\%29}{$Ei(x)$ на Wolfram Alpha} 
	\includegraphics[scale=0.4]{pics/qr_ei_wolfram.eps} 
	}

Часто для проведения расчетов, особенно с использованием компьютерных библиотеке расчетов, бывает удобнее пользоваться модифицированной интегральной показательной функцией $Ei_1(x)$ или $E_1(x)$ или $Ei_n(x)$ при $n=1$.
 $$Ei_n(x) = \int\limits_{1}^{\infty}\frac{e^{-tx}}{t^n}\,\mathrm dt $$

График интегральной показательной функции $Ei_1(x)$ приведен на рисунке \ref{ris:ei1}.
Для вещественных положительных $x\in\mathbb R, x>0$ верно $E_1(x) = - Ei( -x)$

Функцию интегральной экспоненты можно представить в виде ряда. 

$$Ei(x)=-\int\limits_{x}^{\infty}\frac{e^{-t}}{t}\,\mathrm dt=\gamma+\operatorname{ln}|-x|+\sum\limits_{n\ge1}\frac{{-x}^n}{n!\cdot n}, \;  x\in\mathbb R,\;$$

\begin{wrapfigure}{r}{0.5\textwidth}
	%\begin{figure}[h!]
	\begin{center}
		\begin{tikzpicture}
			\begin{axis}
				[axis lines = left,
				width = 0.48\textwidth,
				%xlabel=$x$,
				%ylabel={$f(x)$},
				]
				\addplot gnuplot[no markers, samples=100, domain = 0:2]{expint(1,x)};
				\addlegendentry{$Ei_1(x)$}
				\addplot gnuplot[no markers, samples=100, domain = 0:2]{-log(x)-0.5772};
				\addlegendentry{$ln(x)$}
			\end{axis}
		\end{tikzpicture}
		\caption{Сравнение функций интегральной экспоненты $E_1(x)$ и $ln(x)$.}
		\label{ris:ei2}
	\end{center}
	%\end{figure}
\end{wrapfigure}

Из приведенного выражения можно сделать выводы, что для маленьких значений аргумента  функция интегральной экспоненты $E_1(x)$ может быть аппроксимирована логарифмической зависимостью. 

$$E_1(x) = -ln(x) - \gamma $$

График сравнения функций $E_1(x)$ и $ln(x)$ показан на рисунке \ref{ris:ei2}. Видно, что хорошей аппроксимация будет только для маленьких значений аргумента $x < 0.01$. Но для решения уравнения фильтрации именно эта зона представляет наибольший интерес.

%\begin{wrapfigure}{r}{0.5\textwidth}
\begin{figure}[h!]
	\begin{center}
		\begin{tikzpicture}
			\begin{axis}
				[axis lines = left,
				width = 0.98\textwidth,
				xlabel=$x$,
				ylabel={$f(x)$},
				xmode=log,
				log ticks with fixed point,
				]
				\addplot gnuplot[no markers, samples=100, domain = 0.00001:20 ]{expint(1,x)};
				\addlegendentry{$Ei_1(x)$}
				\addplot gnuplot[no markers, samples=100, domain = 0.00001:20]{-log(x)-0.5772};
				\addlegendentry{$ln(x)$}
			\end{axis}
		\end{tikzpicture}
		\caption{Сравнение функций интегральной экспоненты $E_1(x)$ и $ln(x)$ в логарифмическом масштабе.Можно оценить диапазон применимости логарифмической аппроксимации.}
		\label{ris:ei3}
	\end{center}
\end{figure}
%\end{wrapfigure}

Представление интегральной экспоненты в виде логарифмической аппроксимации удобно на практике, так как логарифм легче вычислять. В большинстве языков программирования и инструментов для проведения расчетов расчет логарифма реализован по умолчанию. А для расчета интегральной экспоненты, часто приходится предпринимать дополнительные шаги.

Решение уравнения фильтрации для линейного стока с учетом логарифмической аппроксимации можно представить в виде 
$$ p_D(r_D,t_D) = \frac{1}{2} \left( ln \left( \dfrac{ t_D }{r_D^2}  \right) +0.809 \right) $$

при использовании данного уравнения, следует помнить, что приближенное решение применимо при $\dfrac{r_D^2}{4t_D} < 0.01$

Решение линейного стока в размерных переменных
$$ p\left(r,t\right)=p_i-\frac{18.41q_sB\mu}{kh}\left(-\frac{1}{2}Ei\left(-\frac{\varphi\mu c_tr^2}{0.00144kt}\right)\right) $$

Решение с учетом логарифмической аппроксимации в размерных переменных
$$ p\left(r,t\right)=p_i-\frac{9.205q_sB\mu}{kh}\left(ln{\frac{kt}{\varphi\mu c_tr^2}}-7.12\right)$$
верно при 
$$\frac{kt}{\varphi\mu c_tr^2}>70000 $$

Решения приведены для практических метрических единиц измерения, что можно увидеть по размерному коэффициенту. 

Скин фактор и нестационарное решение
$$ P(r, t) = P{t} - \frac {9.205\mu {q_s} B }{k h}(\ ln\frac {k t}{ \phi \mu {c_t} {r^2}} +7.12 + 2S) $$


%\subsubsection{Радиус исследования}
%Надо бы тут описать концепцию радиуса исследований и подходы к его оценке



\subsubsection{Решение для конечного радиуса скважины}
Для получения сложных решений уравнения фильтрации часто используется преобразование Лапласа.

$$ L \left [ f(t) \right] = \tilde{f}(s) = \int_{0}^{\infty}f(t)e^{-st}dt $$

где $s$ параметр пространства Лапласа соответствующий времени

Решение для бесконечно малого радиуса скважины в пространстве Лапласа будет иметь вид

$$ \tilde{p}_D(s) = \frac{1}{s} K_0 \left( r_D \sqrt s  \right) $$

решение для конечного радиуса скважины

$$ \tilde{p}_D(s) = \frac{K_0 \left( r_D \sqrt{s}  \right) }{ s \sqrt{s} K_1 \left( \sqrt s  \right)  } $$

где 

$K_0$, $K_1$ - модифицированные функции Бесселя 

Перевод решения из пространства Лапласа в обычное пространство не всегда возможен аналитически. В современных условиях перевод делается численно с использованием компьютеров, что позволяет строить и исследования решения уравнения фильтрации при различных условиях. 

Широко распространено применения алгоритма Стефеста для численного обратного преобразования Лапласа. 