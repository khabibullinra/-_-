\section{Термометрические исследования и расчет распределения температуры потока в скважине и трубопроводах}

\subsection{Теория}
\begin{itemize}
    \item Теория по замеру температуры в стволе скважины и в потоке
    \item Принципы измерения температуры и датчики температуры
\end{itemize}


\subsection{Методы исследования с использованием измерений температуры}

\begin{itemize}
    \item Термометрия в стволе скважины. Расчет температуры на устье. Выявление зон опасных для отложения парафинов гидратов
    
    \item Термометрия в продуктивном интервале 
    
    \item Термометрия в газлифтной скважине для анализа работы газлифтных клапанов
    
    \item Температурный режим работы ЭЦН. Предотвращение отказов ЭЦН на основе замеров температуры
\end{itemize}

\subsection{Задачи}

Задачи с использованием Excel и пакетов программ. 

\begin{itemize}
    \item Расчет распределения температуры в скважине с учетом притока разных флюидов в скважине
    \item Калориметрическое смешивание
    \item Расчет распределения температуры в газлифтной скважине
    \item Расчет температурного режима работы ЭЦН
    \item Расчет температуры и отложение гидратов
    
\end{itemize}
