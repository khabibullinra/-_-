\section{Теория}
\begin{itemize}
    \item Связь давления и движения флюидов в трубах и скважине. Формула Дарси-Вейсбах, уравнение Навье Стокса. Важность учета PVT при моделировании потоков
    \item Методы расчета распределения давления в потоке. Краткой обзор методов и программ
    \item Принципы измерения давления, температурная компенсация, Приборы для измерения давления в эксплуатируемой скважине
    \item Забойное давление - ключевой параметр для контроля работы скважины. Методы контроля забойного давления - прямые измерения и косвенные оценки для различных типов скважин. Частота измерения забойного давления. Цели и задачи измерения забойного давления.
    \item хранение результатов измерений давления. Шахматка, Техрежим, БСИ ЭЦН, корпоративные базы данных.
\end{itemize}

\section{Задачи}

Задачи с использованием Excel и пакетов программ. 

\section{Задачи}
\begin{itemize}
    \item Расчет распределения температуры в скважине с учетом притока разных флюидов в скважине
    \item Калориметрическое смешивание
    
\end{itemize}
