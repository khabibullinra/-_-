\section{Барометрические исследования и расчет распределения давления в скважине и трубопроводах}

Давление - ключевой параметр в системе пласт - скважина - скважинное оборудование. Перепад давления вызывает движение флюидов. Давлением можно управлять с помощью оборудования. Поэтому замеры давления и исследования построенные на контроле давления - ключевые для нефтедобычи.

Поскольку нефть и пластовые флюиды транспортируются по трубам - то изучение многофазного потока в трубах и скважинах занимает заметное место в управлении добычей. На моделировании этих процессов основан ряд методов исследования. 

В этом разделе посмотрим некоторые подходы к моделированию систем нефтедобычи и проведению исследований этих систем связанных с замерами давления.

\subsection{Теория}
\begin{itemize}
    \item Измерение давлений - датчики и принципы. Краткий обзор и ссылки. Принципы измерения давления, температурная компенсация, Приборы для измерения давления в эксплуатируемой скважине
    \item Связь давления и движения флюидов в трубах и скважине. Формула Дарси-Вейсбаха, уравнение Навье Стокса. Важность учета PVT при моделировании потоков
    \item Методы расчета распределения давления в потоке. Краткой обзор методов и программ. Примеры расчета.
    \item 
\end{itemize}

\subsection{Методы исследования с использованием измерений давления}
\begin{itemize}
    \item Забойное давление - ключевой параметр для контроля работы скважины. Методы контроля забойного давления - прямые измерения и косвенные оценки для различных типов скважин. Частота измерения забойного давления. Цели и задачи измерения забойного давления.
    \item Хранение результатов измерений давления. Шахматка, Техрежим, БСИ ЭЦН, корпоративные базы данных.
    \item Барометрия в работающей скважине
    \item Оценка параметров работы заглушенных скважин
\end{itemize}

\subsection{Задачи}

Задачи с использованием Excel и пакетов программ. 

\begin{itemize}
    \item Расчет распределения давления в скважине с использованием гидравлических корреляций. Построение кривой распределения давления в скважине.
    \item Расчет забойного давления по динамическому уровню в скважине с ЭЦН. Анализ отжима динамического уровня. 
    \item Оценка пластового давления в заглушенной скважине
\end{itemize}
