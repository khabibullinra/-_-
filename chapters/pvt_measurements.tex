\section{Исследования физико-химических свойств пластовых флюидов}

Здесь основные определения и теоретический минимум для последующих разделов. 
И набор мелких "коварных" задач для контроля знаний.
\subsection{Теория и определения}
\begin{itemize}
    \item Пластовые флюиды. Вода, нефть, газ.
    \item Модели флюидов. Модель нелетучей нефти. Композиционная модель. Основные предположения и отличия
    \item Параметры нефти для модели нелетучей нефти. Из зависимость от давления и температуры. 
    \item Параметры газа. z фактор. Коэффициент Джоуля Томсона для углеводородных газов. 
    \item Параметры воды и пара.
    
\end{itemize}
\subsection{Методы определения свойств флюидов}
Отбор глубинных проб, замеры на поверхности. Экспресс методы исследования.
\subsection{Задачи}
Расчеты с использованием Excel, Python, специализированное ПО
\begin{itemize}
    \item Зависимости параметров нефти воды и растворенного газа от давления и температуры (можно с использованием унифлок Excel)
    \item Построить график плотности воды при одновременно изменяющихся давлении и температуре
    \item Построить график зависимости доли свободного газа  от давления и температуры в потоке
    \item Оценить изменение температуры газа и ГЖС при изменении давления из за эффекта Джоуля Томсона
    \item Рассчитать изменение свойств воды и пара при нагреве.
    \item оценка газового фактора по данным замера расхода газа и нефти - учет растворенного в нефти газа на замерной с высоким давлением.

    % недавно было обнаружено что график достаточно странно выглядит :) 
    %\item Расчеты закона Дарси для трубы и простые задачи по линейному потоку в пласте. Определить перепад давления в трубе, оценить скорость движения жидкости и так далее.
\end{itemize}
    