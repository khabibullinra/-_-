
\section{Задачи по простым решениям уравнения фильтрации}

\subsection{Динамика забойного давления добывающей скважины}

\subsubsection{Условие задачи}

В бесконечном однородном пласте запускается добывающая скважина с различными дебитами. 
Постройте график изменения забойного давления во времени в процессе работы скважины.

Исходные данные для построения выберите произвольно из диапазоне указанного в таблице \ref{table:data1}, значения дебитов после запуска скважины из таблицы \ref{table:data2}. 



\begin{table}[h!]
%	\ttabbox
%	\centering
	{
		\caption{Исходные данные для задачи}
		\label{table:data1}
	}
	{
		\begin{tabular}{|l|c|l|l|}
			\hline
		\rowcolor{lightgray}	Параметр& Обозначение &Диапазон &Размерность \\
			\hline
			Пластовое давление&  $P_{res} $&$ 150 \dots 250$ &атм \\
			\hline
			Проницаемость пласта &$k $&$ 1 \dots 100$& мД  \\
			\hline
			Эффективная мощность & $h $&$ 2\dots50$ &м  \\
			\hline
			Пористость & $\varphi $&$ 0.1 \dots 0.25$&  \\
			\hline
			Общая сжимаемость& $c_{t} $&$ 2\cdot 10^{-5} \dots2\cdot 10^{-4} $ & 1/атм  \\
			\hline
			Радиус скважины &$r_{w} $&$ 0.1\dots0.2 $& м  \\
			\hline
			Вязкость нефти& $\mu $&$ 0.2\dots2$& сП  \\
			\hline
			Объемный коэффициент   &$B $&$ 1.05\dots 1.3$& м$^3$/м$^3$  \\
			\hline
		\end{tabular}
	
	}
\end{table}

%\floatsetup[table]{capposition=bottom}

\begin{figure}[h!]
	\centering
	\begin{minipage}{0.45\textwidth}
		\centering
			\begin{tabular}{|c|c|}
				\hline
				\rowcolor{lightgray}	Время, часы& Дебит, м$^3$/сут  \\
				\hline
				0 & 10  \\
				\hline
				10 & 20  \\
				\hline
				30 & 30  \\
				\hline
				56 & 0  \\
				\hline
				256 & 0 \\
				\hline
			\end{tabular}
			
	\end{minipage}
	\hfill
	\begin{minipage}{0.45\textwidth}
		\centering
		\begin{tikzpicture}
			\begin{axis}
				[axis lines = left,
				width = \textwidth,
				xlabel={$t$ часы},
				ylabel={$Q$ м$^3$/сут} 
				],
				\addplot+[const plot]
				coordinates
				{(0,0) (0,10)    (10,20)  (30,30)   (56,0) (256,0)};
			\end{axis}
		\end{tikzpicture}
	\end{minipage}

	\caption{Пример значений дебитов после запуска.} \label{fig:mult2}

	
\end{figure}

Для проверки корректности расчета ваш метод расчета должен позволять быстро провести расчеты для любых значений параметров из заданного диапазона значений и для любых значений дебитов. Для этого рекомендуется создать расчетный модуль в Excel, который позволил бы быстро провести расчет по введенным исходным данным.

\subsubsection{Дополнительные вопросы к задача}

Для того, чтобы лучше разобраться с задачей можно попробовать ответить на несколько дополнительных вопросов.
\begin{itemize}
	\item Какие модели можно использовать для решения задачи? Сколько различных вариантов расчетов вы можете сделать? 
	\item При каких условиях решения по разным моделям будут отличаться?
	\item При каких условиях необходимо использовать модель с учетом конечного радиуса скважины?
	\item Если история дебитов большая - то как это влияет на скорость расчета? Как можно увеличить скорость расчета?
\end{itemize}

\subsubsection{Решение}
	Надо использовать принцип суперпозиции.
	
	Более детальное описание и подсказки будут позже тут.
	
	Варианты решения -- макросы унифлок, программа типа Каппа, python и jupyter notebook.
	
\subsection{Сравнение аналитического и численных решений}

\subsubsection{Условие задачи}
Постройте график восстановления давления на скважине после остановки с использованием аналитической зависимости и с использованием гидродинамического симулятора. Сравните полученные решения - стационарное аналитическое решение, нестационарное аналитическое решение, численное решение. Проведите исследование - какие параметры моделей влияют на различия построенных решений.

\subsubsection{Решение}
Можно использовать симуляторы RSO, OPM или любой другой доступный симулятор.


\subsection{Динамика изменения дебита добывающей скважины}
\subsubsection{Условие задачи}
Постройте график зависимости дебита добывающей скважины после запуска с постоянным забойным давлением


\subsubsection{Дополнительные вопросы к задаче}

Для того, чтобы лучше разобраться с задачей можно попробовать ответить на несколько дополнительных вопросов.
\begin{itemize}
	\item Как будет выглядеть решение для случая, когда забойное давление и дебит связаны линейным соотношением (модель запуска скважины с ЭЦН)? 
	\item Динамическая индикаторная кривая и ее сравнение со стационарной индикаторной кривой?

\end{itemize}