
\section{Интерпретация исследования на неустановившемся режиме работы}

В ходе исследования на неустановившемся режиме работы в скважине регистрируется ряд показателей во времени, на основе которых проводится интерпретация -- определяются параметры пласта и скважины. Минимальный набор регистрируемых параметров - дебит скважины $q_{liq}$ и  забойное давление $p_{wf}$. Иногда регистрируется более широкий набор -- устьевые давления и температура, забойная температура, динамический уровень. Как правило больший набор параметров позволяется получить более надежные результаты, даже если интерпретация будет основана только на забойном давлении. Дополнительные параметры позволяют убедиться, что исследование проведено по плану и никакие другие факторы на результат на повлияли.

Наиболее ценными данными для ГДИС является динамика изменения забойного давления. Забойное давление регистрировать значительно проще чем дебит. Поэтому стараются организовать исследование таким образом, чтобы изменений  дебита была как можно меньше, и он менялся резко, за короткий промежуток времени.Например при тесте на восстановление давления КВД скважину работающую со стабильным дебитом $q_{liq}$ останавливают $q_{liq} = 0$) при этом забойное давление регистрируется с большой частотой. Иногда, при невозможности замерить забойное давление напрямую, регистрируют параметры, позволяющие оценить забойное давление - например динамический уровень и затрубное давление. Значения динамики изменения дебитов и забойных давлений фиксируют и далее интерпретируют.

Суть интерпретации сводится к поиску модели работы пласта и скважины и параметров этой модели, которые позволили бы описать наблюдаемые изменения параметров. Если такая модель будет найдена, то можно предположить, что в течении некоторого периода времени после проведения исследования она будет корректно описывать работу скважины, а значит с ее помощью можно будет построить прогноз работы скважины, например после ГТМ, что позволит оценить эффективность ГТМ и принять решение о целесообразности его проведения.

При проведении интерпретации можно выделить следующие этапы:
\begin{enumerate}
    \item Предварительная обработка данных исследования. Выделение целевого интервала, при необходимости фильтрация данных, удаление выбросов, корректировки времени замеров разных датчиков и т.д.
    \item Выбор типа модели для описания замеров. Как правило проводится с использованием диагностического графика в двойных логарифмических координатах, позволяющего идентифицировать характерные признаки различных моделей.
    \item Определение параметров модели по измеренным данным. Часто этот процесс называют "матчингом" (от англиского to match), фиттингом (to fit) или настройкой модели. Именно этот этап обычно называют интерпретацией и подробно описывают в книгах и программном обеспечении. Также этот этап достаточно легко автоматизируется, например с использованием алгоритмов нелинейной регрессии. Иногда результат интерпретации выражается в виде доверительных интервалов - вместо одно числа скин-фактора или проницаемости указывает диапазон значений в которых параметры могут лежать с заданной вероятностью.
    \item Анализ результатов интерпретации. Обычно, не всегда удается добиться хорошего совпадения замеров и модели, в таких случаях инженер должен определить качество интерпретации используя для этого всю доступную информацию о пласте и скважине, ограничениях модели, целях исследования. Интерпретацию можно признать качественной, если она позволяет решить поставленную задачу, с учетом всей имеющейся информации.  
\end{enumerate}

Часто этапы интерпретации зацикливаются - проверяются гипотезы о возможности описания измерений разными моделями, при этом для каждой модели проводится определение параметров и анализ результатов. 

Выбор модели, а также анализ результатов интерпретации являются наиболее сложными этапами, требующими знания особенностей и ограничения различных моделей интерпретации.


\section{Радиальный приток. Метод прямых линий.}

\subsection{Тест на падение давления на добывающей скважине}

При тесте на падение давления (КПД) в добывающей скважине мы запускаем скважину с постоянным дебитом и наблюдаем за тем как изменяется забойное давление.

Для случая радиального притока к вертикальной скважине в однородном пласте изменение давления на забое скважины запущенной с постоянным дебитом можно описать с использованием логарифмической аппроксимации решения линейного стока.

\begin{equation}
 p\left(r,t\right)=p_i-\frac{9.205q_s B\mu}{kh} \left(ln{\frac{kt}{\varphi\mu c_t r^2}}-7.12 +2S\right)    
\end{equation}

которое верно при 
$$\frac{kt}{\varphi\mu c_tr^2}>70000 $$

$p(r,t)$ - давление в пласте на расстоянии $r$, м. от скважины в момент времени $t$ час. измеряется в атм.

$p_i$ - начальное давление (pressure initial) атм

$q_s$ - дебит жидкости на поверхности (surface rate), м$^3$/сут 

$B$ - объемный коэффициент нефти, м$^3$/м$^3$

$\mu$ - вязкость нефти, сП

$k$ - проницаемость, мД

$h$ - эффективная можность пласта, м

$\varphi$ - пористость, д.е.

$c_t$ - общая сжимаемость породы и флюида, 1/атм

$S$ - скин-фактор


Это одно из самых простых решений для описания динамики изменения забойного давления. Кроме того это решение хорошо стыкуется с широко распространенным стационарным решением -- формулой Дюпюи. Поэтому оценив проницаемость и скин по логарифмической аппроксимации - можно использовать их для построения индикаторной кривой скважины на основе формулы Дюпюи и для оценки потенциала скважины и ожидаемых эффектов от ГТМ. 

Эти соображения приводят к тому, что при проведении любых исследований на неустановившемся режиме - первая гипотеза которая проверяется - гипотеза о наличии радиального притока к скважине и оценка параметров радиального притока.  
Для проверки этой гипотезы уравнение можно преобразовать 

$$ p_{wf}\left(t\right)=p_i-\frac{9.205q_sB\mu}{kh}\left( ln{t} + ln{\frac{k}{\varphi\mu c_tr^2} }-7.12 + 2S\right)$$

Можно увидеть, что в данном случае изменение давления $p_i p_{wf} = m_{ln} ln(t) + n_s$, где $m_{ln} = const, n_{ln}=const$  линейная функция логарифма времени. Следовательно нарисовав изменение давление в полулогарифмических координатах ($p$ от $\ln{t}$) зависимость будет иметь вид прямой линии, которую легко идентифицировать визуально (или при помощи компьютера, например построив линию тренда в Excel).


В реальности вид динамики изменения забойного давления в полулогарифмических координатах может отличаться от прямой линии. Отклонения от радиального притока могут возникать из за различных факторов. Основные - влияния послепритока (на начальных временах), влияние границ (на поздних временах), шум датчика, влияние трещины ГРП или горизонтального ствола. 
Из за отклонений выделение участка радиального притока может оказаться сложным. Часто прямолинейные участки могут быть выделены ошибочно.  Поэтому метод прямых линий можно рассматривать как вспомогательный. 
Если прямую линию соответствующую радиальному притоку удастся выделить, тогда ее можно использовать для восстановления параметров пласта (метод полулогарифмической анамарфозы, метод МДХ - Миллера Дайеса Хатчинсона). 

Идея метода состоит в том, чтобы нарисовать изменение давления в полулогарифмических координатах и для участка соответствующего радиальному притоку к скважине построить прямую линию совмещенную с нашими данными.


Определив наклон линии $m_{ln}$  можно найти проницаемость и скин

$$ k=9.205\frac{q_s B \mu}{m_{ln} h} $$

$$ S = \frac{1}{2} \left[ \frac{ p_i - p_{t=1} }{m_{ln}} - ln{ \frac{k}{\phi \mu c_t r_w^2} +7.12}  \right]$$

где $p_{t=1}$ давление по модели радиального притока соответствующее моменту времени $t=1$ (пересечение прямой линии которую мы подобрали с вертикальной линией соответствующей $t=1$ ).


\subsection{Тест на восстановление давления на добывающей скважине}

Здесь надо показать как трансформируется решение с учетом принципа суперпозиции - время Хорнера и тому подобное.

\section{Диагностический график}

Диагностический график - определение и свойства. Логарифмическая производная 

\subsection{Признаки радиального притока на диагностическом графике}
Удобный способ анализа динамических данных  по изменению давления - построение данных $\Delta p$ от $ln(t)$


Обратим внимание, что логарифмическая производная от изменения давления для участка радиального притока будет выглядеть как 

$$ \frac{dP}{d ln{t}} = m_{ln} $$

Радиальный приток для теста на восстановление давления

\subsection{Признаки влияния ствола скважины на диагностическом графике}

Показать что линия с единичным наклоном хорошо описывает постоянный послеприток

\subsection{Признаки линейного притока на диагностическом графике}

\section{Метод нелинейной регрессии}

Показать как можно обеспечить оптимальное совпадение замерных данных и модели с использованием алгоритмов нелинейной регрессии (поиска решений в Excel)