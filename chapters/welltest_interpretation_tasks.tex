\section{Задачи по интерпретации ГДИС}

\subsection{Простое исследование на неустановившемся режиме. Тест на падение давления на добывающей скважине.}

\subsubsection{Условие задачи}
Задана добывающая скважина, работающая чистой нефтью, со следующими параметрами (приведены в таблице \ref{table:test_data1})).

\begin{table}[h!]
	%\ttabbox
	%	\centering
	{
		\caption{Исходные данные для задачи}
		\label{table:test_data1}
	}
	{
		\begin{tabular}{|l|c|c|l|}
			\hline
			\rowcolor{lightgray}	Параметр& Обозначение &Диапазон &Размерность \\
			\hline
			Радиус скважины&  $r_{w} $&$ 0.1 $ &м \\
			\hline
			Эффективная мощность & $h $&$ 10$ &м  \\
			\hline
			Пористость & $\varphi $&$ 0.15 $&  \\
			\hline
			Общая сжимаемость& $c_{t} $&$ 4\cdot 10^{-5}  $ & 1/бар  \\
			\hline
			Вязкость нефти& $\mu $&$ 0.8$& сП  \\
			\hline
			Объемный коэффициент   &$B $&$ 1.2$& м$^3$/м$^3$  \\
			\hline
			Пластовое давление   &$P_{res} $&$ 250$& атм  \\
			\hline
		\end{tabular}
		
	}
\end{table}

Скважина в течении длительного времени работала с дебитом $Q_{oil} = 80$ м$^3$/сут. Затем была остановлена на КВД. После остановки проводился замер динамики изменения забойного давления, результаты замеров приведены на рисунке \ref{ris:test01}. Исходные данные можно скачать в приложении.


\begin{figure}[h]
	\caption{Измеренные данные для задачи 1. График от времени}
	\label{ris:test01}		
	\begin{tikzpicture}
		\begin{axis}[
			width=\textwidth,
			height = 0.5 \textwidth,
			ylabel={давление, бар}]
			\addplot[color=blue, mark=*] table [y=Pressure, x=Time]{data/test01/pressure.txt};
			\legend{замеры давления}
		\end{axis}
	
	\end{tikzpicture}
	\begin{tikzpicture}
		\begin{axis}[
			width=\textwidth,
			height = 0.3 \textwidth,
			xlabel={время, часы},
			ylabel={дебит, м$^3$/сут},
			legend pos=south east]
			\addplot[color=red, mark=*]  table [y=LiquidRate, x=Time]{data/test01/rates.txt};
			\legend{замеры дебита}
		\end{axis}
	\end{tikzpicture}
\end{figure}

% здесь S=6, k = 30 мД, Cs = 0.1 m3/bar

Оцените величины проницаемости пласта $k$, скин-фактора $S$ и коэффициента послепритока $C_s$. Постройте график демонстрирующий совпадение кривых исходных и модельных с подобранными параметрами в обычных, полулогарифмических и двойных логарифмических координатах. На графике в двойных логарифмических координатах также отобразите графики логарифмической производной. Отобразите на графиках характерные линии для радиального притока и послепритока.


\subsubsection{Рекомендация - дополнительное задание}

Для выполнения расчетов удобно создать расчетный модуль в Excel в который можно было бы загрузить исходные данные и быстро получить ответ.

Ответ относительно легко можно получить двумя способами - методом прямых линий (метод МДХ) и с использованием нелинейной регрессии (используйте поиск решения в Excel).


\subsection{Простое исследование на неустановившемся режиме. Тест на восстановление давления на добывающей скважине.}

\subsubsection{Условие задачи}
Задана добывающая скважина, работающая чистой нефтью, со следующими параметрами (приведены в таблице \ref{table:test_data2})).

\begin{table}[h!]
	%\ttabbox
	%	\centering
	{
		\caption{Исходные данные для задачи}
		\label{table:test_data2}
	}
	{
		\begin{tabular}{|l|c|c|l|}
			\hline
			\rowcolor{lightgray}	Параметр& Обозначение &Диапазон &Размерность \\
			\hline
			Радиус скважины&  $r_{w} $&$ 0.1 $ &м \\
			\hline
			Эффективная мощность & $h $&$ 10$ &м  \\
			\hline
			Пористость & $\varphi $&$ 0.15 $&  \\
			\hline
			Общая сжимаемость& $c_{t} $&$ 4\cdot 10^{-5}  $ & 1/бар  \\
			\hline
			Вязкость нефти& $\mu $&$ 0.8$& сП  \\
			\hline
			Объемный коэффициент   &$B $&$ 1.2$& м$^3$/м$^3$  \\
			\hline
		\end{tabular}
		
	}
\end{table}

Скважина работала более 1000 часов со стабильным дебитом, затем была остановлена на 2 суток. За это время были записаны данные по изменению забойного давления (рисунок \ref{ris:test02}). По данным изменения забойного давления во время остановки оцените величины проницаемости пласта $k$, скин-фактора $S$ и коэффициента послепритока $C_s$ и пластового давления $P_{res} $. Постройте график демонстрирующий совпадение кривых исходных и модельных с подобранными параметрами в обычных, полулогарифмических и двойных логарифмических координатах. На графике в двойных логарифмических координатах также отобразите графики логарифмической производной. Отобразите на графиках характерные линии для радиального притока и послепритока.

\begin{figure}[h]
	\caption{Измеренные данные для задачи 2. График от времени}
	\label{ris:test02}		
	\begin{tikzpicture}
		\begin{axis}[
			x tick label style={/pgf/number format/.cd,%
									scaled y ticks = false,
									set thousands separator={},
									fixed},
			width=\textwidth,
			height = 0.5 \textwidth,
			ylabel={давление, бар}]
			\addplot[color=blue, mark=*] table [y=Pressure, x=Time]{data/test02/pressure.txt};
			\addlegendentry{замеры давления}
		\end{axis}
		
	\end{tikzpicture}
	\begin{tikzpicture}
		\begin{axis}[
			x tick label style={/pgf/number format/.cd,%
				scaled y ticks = false,
				set thousands separator={},
				fixed},
			width=\textwidth,
			height = 0.3 \textwidth,
			xlabel={время, часы},
			ylabel={дебит, м$^3$/сут},
			legend pos=south east]
			\addplot[color=red, mark=*]  table [y=LiquidRate, x=Time]{data/test02/rates.txt};
			\addlegendentry{замеры дебита}
		\end{axis}
	\end{tikzpicture}
\end{figure}