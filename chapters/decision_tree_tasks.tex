\section{Задачи с примерами решения}

Деревья решений являются удобным инструментом для формализации процесса принятия решений. Они достаточно широко используются в нефтяной индустрии [привести ссылки на книгу и статьи]. 

Для построения деревьев решений можно использовать ручку, бумагу и калькулятор, так расчеты достаточно просты. Но удобнее использовать специализированные программные инструменты. Далее мы будем использовать сайт \url{http://silverdecisions.pl/} предоставляющий удобный, и что не менее важно, бесплатный и открытый программный комплекс для построения и анализа деревьев решений.

\subsection{Исследования на разведочной скважине}
    
    
Оцените стоит ли проводить комплекс исследований на разведочной скважине со следующими параметрами:
\begin{itemize}
    \item Стоимость 1 млн руб
    \item Ожидаемая информация – фильтрационные параметры пласта, уточнение строение пласта
    \item Вероятность успешности исследования (получения какой то информации) 70%
    \item Ожидается что исследование подтвердит увеличение запасов на 15% (60% что запасы увеличатся)
    \item Текущие извлекаемые запасы – 1 млн т. нефти, стоимость 1 т.нефти в запасах – 1 тыс. руб.
\end{itemize}     
Стоит ли проводить исследование? 

    
\subsection{КВД на добывающей скважине}

Оцените стоит ли проводить КВД на эксплуатационной скважине с дебитом 50 т/нефти, Рзаб = 50 атм, Рпл = 250 атм
\begin{itemize}
    \item Стоимость исследования 1 млн руб
    \item Длительность 1 неделя (скважина остановлена)
    \item Ожидаемая информация – фильтрационные параметры пласта, скин фактор
    \item Вероятность успешности исследования (получения какой то информации) 70%
    \item Ожидается что исследование подтвердит наличие положительного скин фактора
    \begin{itemize}
        \item S=0 с вероятностью 50%
        \item S=5 с вероятностью 40%
        \item S=15 с вероятностью 10%
    \end{itemize}
    \item Стоимость кислотной обработки снижающей скин в 2 раза 1 млн руб. Длительность эффекта 1 год
    \item Стоимость нефти 1 т = 10 тыс. руб.
\end{itemize}
Стоит ли проводить исследование? 

\subsection{Исследование фонтанирующей скважины}
Имеется фонтанирующая скважина. Дебит нефти 30 т/сут. Воды нет. Имеется водоносный пласт под продуктивным. Перемычка 10 м. Вероятность получения прорыва воды после ГРП 50 \%. При прорыве воды рост обводненности до 90 \% без увеличения дебита нефти. Без прорыва воды увеличение продуктивности в 3 раза. Стоимость ГРП 1 млн. руб.

Оценить что делать на скважине?

Ничего не делать
Сделать ГРП
Провести исследование и по его результатам ГРП
Стоимость исследования 100 т.р.

Вероятность выявления скважины где ГРП будет успешен 60%