\newcommand{\university}{РГУ нефти и газа (НИУ) имени И.М.Губкина}
\newcommand{\faculty}{кафедра разработки и эксплуатации нефтяных месторождений}
\newcommand{\department}{департамент}
\newcommand{\city}{Москва}
\newcommand{\num}{ № 1}
\newcommand{\docname}{Исследование скважин и пластов. Введение}
\newcommand{\tutorname}{Хабибуллин Р.А.}
\newcommand{\studentname}{студент}
\newcommand{\group}{группы нет}


% настройка кодировки, шрифтов и русского языка
\usepackage{fontspec}
\usepackage{polyglossia}

% рабочие ссылки в документе
\usepackage{hyperref}
\urlstyle{same}
% графика
\usepackage{graphicx}

% качественные листинги кода
%\usepackage{minted}
%\usepackage{listings}
%\usepackage{lstfiracode}

% отключение копирования номеров строк из листинга, работает не во всех просмотрщиках (в Adobe Reader работает)
\usepackage{accsupp}
\newcommand\emptyaccsupp[1]{\BeginAccSupp{ActualText={}}#1\EndAccSupp{}}
\let\theHFancyVerbLine\theFancyVerbLine
\def\theFancyVerbLine{\rmfamily\tiny\emptyaccsupp{\arabic{FancyVerbLine}}}

% библиография
\bibliographystyle{templates/gost-numeric.bbx}
\usepackage{csquotes}
\usepackage[parentracker=true,backend=biber,hyperref=true,bibencoding=utf8,style=numeric-comp,language=auto,autolang=other,citestyle=gost-numeric,defernumbers=true,bibstyle=gost-numeric,sorting=ntvy]{biblatex}

% для заголовков
\usepackage{caption} 

% разное для математики
\usepackage{amsmath, amsfonts, amssymb, amsthm, mathtools}

% водяной знак на документе, см. main.tex
%\usepackage[printwatermark]{xwatermark} 

% для презентаций
\usepackage{here}
\usepackage{animate}
\usepackage{bm}


% настройки polyglossia
\setdefaultlanguage{russian}
\setotherlanguage{english}

% локализация
\graphicspath{ {} }
\addto\captionsrussian{
  \renewcommand{\partname}{Глава}
  \renewcommand{\contentsname}{Содержание}
  \renewcommand{\figurename}{Рисунок}
%  \renewcommand{\listingscaption}{Листинг}
}

% основной шрифт документа
\setmainfont{CMU Serif}

% перечень использованных источников
\addbibresource{refs.bib}

% оформление презентации
\usetheme{metropolis}
\usecolortheme{seagull}
\beamertemplatenavigationsymbolsempty

% настройка ссылок и метаданных документа
\hypersetup{unicode=true,colorlinks=true,linkcolor=red,citecolor=green,filecolor=magenta,urlcolor=cyan,        		       
    pdftitle={\docname},   	    
    pdfauthor={\studentname},      
    pdfsubject={\docname},         
    pdfcreator={\studentname}, 	       
    pdfproducer={Overleaf}, 		     
    pdfkeywords={\docname}
}

% настройка подсветки кода и окружения для листингов
%\usemintedstyle{colorful}
%\newenvironment{code}{\captionsetup{type=listing}}{}

% шрифт для листингов с лигатурами
%\setmonofont{FiraCode-Regular.otf}[
%    Path = templates/,
%    Contextuals=Alternate
%]

% путь к каталогу с рисунками
\graphicspath{{fig/}}

% настоящее матожидание
\newcommand{\MExpect}{\mathsf{M}}

% объявили оператор!
\DeclareMathOperator{\sgn}{\mathop{sgn}}