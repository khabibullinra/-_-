


% настройка кодировки, шрифтов и русского языка
\usepackage[russian]{babel}
\usepackage{fontspec}

% оформление презентации
\usetheme{metropolis}
\usecolortheme{seagull}


% рабочие ссылки в документе
\usepackage{hyperref}
\urlstyle{same}
% графика
\usepackage{graphicx}
\usepackage{tikz}
%\usepackage{pgfplots}

% качественные листинги кода
%\usepackage{minted}
%\usepackage{listings}
%\usepackage{lstfiracode}


% библиография
\bibliographystyle{templates/gost-numeric.bbx}
\usepackage{csquotes}
\usepackage[parentracker=true,backend=biber,hyperref=true,bibencoding=utf8,style=numeric-comp,language=auto,autolang=other,citestyle=gost-numeric,defernumbers=true,bibstyle=gost-numeric,sorting=ntvy]{biblatex}

% перечень использованных источников
\addbibresource{refs.bib}


% для заголовков
\usepackage{caption} 

% разное для математики
\usepackage{amsmath, amsfonts, amssymb, amsthm, mathtools}

% водяной знак на документе, см. main.tex
%\usepackage[printwatermark]{xwatermark} 

% для презентаций
\usepackage{here}
\usepackage{animate}
\usepackage{bm}




% локализация
%\graphicspath{ {} }
%\addto\captionsrussian{
%  \renewcommand{\partname}{Глава}
%  \renewcommand{\contentsname}{Содержание}
%  \renewcommand{\figurename}{Рисунок}
%  \renewcommand{\listingscaption}{Листинг}
%}





% настройка ссылок и метаданных документа
\hypersetup{unicode=true,colorlinks=true,linkcolor=red,citecolor=green,filecolor=magenta,urlcolor=cyan,        		       
    pdftitle={\docname},   	    
    pdfauthor={\studentname},      
    pdfsubject={\docname},         
    pdfcreator={\studentname}, 	       
    pdfproducer={Overleaf}, 		     
    pdfkeywords={\docname}
}

% настройка подсветки кода и окружения для листингов
%\usemintedstyle{colorful}
%\newenvironment{code}{\captionsetup{type=listing}}{}

% шрифт для листингов с лигатурами
%\setmonofont{FiraCode-Regular.otf}[
%    Path = templates/,
%    Contextuals=Alternate
%]

% путь к каталогу с рисунками
\graphicspath{{fig/}}

% настоящее матожидание
\newcommand{\MExpect}{\mathsf{M}}

% объявили оператор!
\DeclareMathOperator{\sgn}{\mathop{sgn}}