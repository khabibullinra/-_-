\documentclass{memoir}

\setcounter{secnumdepth}{2}

\usepackage{epigraph}
\usepackage{polyglossia}

\setmainlanguage[babelshorthands=true]{russian}    % Язык по-умолчанию русский с поддержкой приятных команд пакета babel
\setotherlanguage{english}                         % Дополнительный язык = английский (в американской вариации по-умолчанию)

%\setdefaultlanguage{russian}
%\setmainfont{CMU Serif}

\setmonofont{Courier New}                          % моноширинный шрифт
\newfontfamily\cyrillicfonttt{Courier New}         % моноширинный шрифт для кириллицы
\defaultfontfeatures{Ligatures=TeX}                % стандартные лигатуры TeX, замены нескольких дефисов на тире и т. п. Настройки моноширинного шрифта должны идти до этой строки, чтобы при врезках кода программ в коде не применялись лигатуры и замены дефисов
\setmainfont{Times New Roman}                      % Шрифт с засечками
\newfontfamily\cyrillicfont{Times New Roman}       % Шрифт с засечками для кириллицы
\setsansfont{Arial}                                % Шрифт без засечек
\newfontfamily\cyrillicfontsf{Arial}               % Шрифт без засечек для кириллицы

%\setmonofont{LiberationMono}[Scale=0.87]            % моноширинный шрифт
%\newfontfamily\cyrillicfonttt{LiberationMono}[Scale=0.87]   % моноширинный шрифт для кириллицы
    
%\defaultfontfeatures{Ligatures=TeX}                % стандартные лигатуры TeX, замены нескольких дефисов на тире и т. п. Настройки моноширинного шрифта должны идти до этой строки, чтобы при врезках кода программ в коде не применялись лигатуры и замены дефисов
%\setmainfont{LiberationSerif}                      % Шрифт с засечками
%\setmainfont{CMU Serif}
%\newfontfamily\cyrillicfont{LiberationSerif}       % Шрифт с засечками для кириллицы
%\setsansfont{LiberationSans}                       % Шрифт без засечек
%\newfontfamily\cyrillicfontsf{LiberationSans}      % Шрифт без засечек для кириллицы


\usepackage{amsthm}
\usepackage{url} 
\usepackage{hyperref} 


\usepackage[table]{xcolor}
%\let\newfloat\undefined
%\usepackage{floatrow}

% графика
\usepackage{graphicx}
\usepackage{tikz}                   % Продвинутый пакет векторной графики
\usepackage{pgfplots}
\pgfplotsset{compat=newest}
\usepackage{pgfplotstable}

\usepackage{wrapfig}

\usepackage{physics}
\usepackage{amsmath}
\usepackage{tikz}
\usepackage{mathdots}
\usepackage{yhmath}
\usepackage{cancel}
\usepackage{color}
\usepackage{siunitx}
\usepackage{array}
\usepackage{multirow}
\usepackage{amssymb}
\usepackage{gensymb}
\usepackage{tabularx}
\usepackage{booktabs}
\usetikzlibrary{fadings}
\usetikzlibrary{patterns}
\usetikzlibrary{shadows.blur}
\usetikzlibrary{shapes}



\input{biblio/biblatex}

\begin{document}

\title{«Исследование скважин и пластов». Учебное пособие }
%\title{«Применение базовых алгоритмов анализа исследований нефтяных скважин»}
%\title{}

\maketitle

\tableofcontents{}
\chapter{Введение}

\section{Цели и задачи дисциплины}
Учебное пособие подготовлено в рамках курса "Исследования скважин и пластов", читаемого для студентов 4 курса РГУ нефти и газа (НИУ) имени И.М.Губкина и соответствует учебной программе курса по состоянию на 2020 год. 

Пособие описывает некоторые базовые подходы к построению и использованию математических моделей скважин и пластов, широко используемые при проведении и интерпретации исследований с использованием компьютеризированных алгоритмов. Рассматривается широкий диапазон промысловых исследований - от технологических замеров ключевых параметров до внутрискважинных промысловых исследований и исследований пласта и призабойной зоны (во второй части). При этом акцент сделан именно на математических моделях и их реализациях в виде программных алгоритмов, что позволяет получить навыки построения моделей и интерпретации результатов исследований необходимые при широком использовании цифровых технологий в нефтедобычи. В работе широко используются открытые инструменты для проведения инженерных расчетов, что позволяет работать над решением задач на современном уровне без необходимости получения лицензий для коммерческих программ. 

Учебное пособие содержит набор задач, предназначенных для помощи в освоении дисциплины "Исследования скважин и пластов". Авторы уверены, что решение задач - лучший метод для изучения инженерных дисциплин. Поэтому в пособии сделан акцент именно на задачах. Теоретический материал приводится в минимальном виде, необходимом для решения задач. Для более подробного разбора теории приводятся ссылки на соответствующие материалы и статьи. Разобраны базовые примеры решения задач и приведены задачи для самостоятельной проработки в ходе семинаров. 

Пособие предназначено для студентов изучающих вопросы разработки и эксплуатации нефтяных и газовых месторождений, моделирование систем нефтедобычи, а также может быть полезным широкому кругу специалистов применяющих компьютерные математические модели нефтедобычи в практической деятельности. 

Учебное пособие является частью учебно методического комплекса для изучения дисциплины "Исследования скважин и пластов". Часть материалов комплекса доступна в сети интернет на общедоступных ресурсах. Это такие материалы как - программы и макросы для проведения инженерных расчетов, исходные данные для выполнения ряда заданий, видеоролики со вспомогательными материалами. Ссылки на ресурсы приводятся в пособии в состоянии актуальном на момент создания (предприняты все меры, чтобы обеспечить доступность ссылок на максимально длительный период, однако авторы не могут гарантировать работоспособность этих ссылок в длительной перспективе).




\section{Подход к изучению дисциплины Исследования скважин и пластов}
Курс "Исследования скважин и пластов" предназначен для того, чтобы дать студентам представление о современных подходах к проведению исследований на нефтяных месторождениях, а также о методах интерпретации их результатов. Рассматриваются различных виды исследований проводимых на скважинах введенных в эксплуатацию от технологических замеров и промысловых геофизических исследований до гидродинамических исследований.

Технологии проведения исследований и оборудование для их проведения с одной стороны достаточно хорошо описано в классической литературе и в интернет источниках, с другой стороны постоянно эволюционирует и меняется - появляются новые приборы и методы интерпретации, как правило использующие последние достижении техники и компьютерные технологии для интерпретации результатов. Остаются неизменными только физические принципы заложенные в основу исследований и методов их анализа. Понимая это авторы не гонятся за исчерпывающим описанием технологий и оборудования для проведения исследований. Вместо этого пособие описывает базовые принципы различных видов исследований выраженные в виде математических моделей. Причем рассматриваются компьютерные реализации этих моделей, которые могут быть использованы студентами и специалистами для обучения и для решения практических задач. Такие компьютерные модели лежат в основе большинства коммерческих программных комплексов, которые получили распространение в нефтяной промышленности, поэтому понимание основ их работы, сильных и слабых сторон, а также получение навыков построения компьютерных алгоритмов является залогом успешного и эффективного проведения исследования по глубокому убеждению авторов.

Большой объем материала и необходимость скорейшего создания учебного пособия заставили автором разбить пособие на две части. Часть 1 посвящена в большей степени исследованиям скважин - технологическим замерам (замеры дебита) и основам промысловых геофизических исследований (барометрия, термометрия и другие).  Часть 2 посвящена исследованию свойств пласта - гидродинамическим исследованиям скважин и построению простых моделей пласта и скважин - "прокси" моделям. При этом в этой части активно используют и модели скважин, разобранные в части 1.

В пособии рассматриваются принципы проведения и интерпретации распространенных видов исследований, которые лежат в основе многих расчетных методик. Понимание этих принципов и владение навыками проведения расчетов с использованием современных подходов позволит студенту решать широкий круг задач возникающих перед инженерами разработчиками на практике, а также облегчит самостоятельный разбор особенностей отдельных технологий. После разбора задач курса студенты смогут самостоятельно создать простые компьютерные алгоритмы для интерпретации базовых исследований скважин и пластов и при необходимости смогут развить их для обработки больших объемов данных по исследованиям и/или интерпретации нестандартных видов исследований. 

Из за широкого применения компьютерных алгоритмов в пособии читателю (и слушателю курсов) потребуется проявить навыки создания компьютерных программ. В простейшем варианте в виде создания сложных расчетных модулей в электронной таблице Excel. А в более "продвинутом" варианте в виде написания макросов и подпрограмм решающих определенные задачи. Это во многом отличает подход к изучению исследований скважин и пластов от других пособий и курсов известных авторам. Однако, опять же по глубокому убеждению авторов, эти навыки являются крайне полезными и востребованными в современной нефтедобыче, поэтому им уделяется особое внимание. Большинство задач можно решить без написания кода в явном виде (формулы в ячейках Excel по сути тоже являются компьютерным кодом, но здесь мы не имеем их в виду), однако умение создавать простые скрипты часто упрощает решение задач и позволяет применить решения для широкого спектра практически значимых проблем. Пособие не ставит целью научить какому то языку программирования. Подобных материалов много в интернете и ссылки на некоторые будут приведены. Пособие же сосредоточено больше на решении инженерных задач, для некоторых из которых применение элементов программирования может оказаться полезным. 





\chapter{Планирование и проектирование исследований}

Оценка экономической целесообразности проведения исследований. Ответ на вопрос - зачем нужны исследования. 
Будет про данные информацию и модели и описание зачем нужны модели и как правильно их использовать.
Соответственно большинство задач так или иначе посвящено использованию различных моделей. 
Тут будет про деревья решений среди прочего.


\section{Что мы будем понимать под исследованиями?}


\epigraph{Проблемы начинаются не тогда, когда имея в виду одно используют разные слова, а когда используя одни и теже слова имеют в виду разное.}{хорошо бы найти оригинал}

Проведение исследований неразрывно связано с понятиями -- данные, информация, знания. Эти термины всем знакомы, кажутся самодостаточными и очевидными. Их определения и различные трактовки можно легко найти в интернете, например на \url{https://ru.wikipedia.org/}.

\marginpar{
        \href{https://ru.wikipedia.org/wiki/информация}{Информация} 
        \newline
        \includegraphics[scale=0.4]{pics/qr_wikipedia_information.png} 
        \newline
        \href{https://ru.wikipedia.org/wiki/данные}{Данные} 
        \newline
        \includegraphics[scale=0.4]{pics/qr_wikipedia_data.png}
    }

Следствием распространённости этих понятий является многообразие их значений и интерпретаций в различных областях деятельности. Приведём здесь одну из интерпретаций, которой будем в дальнейшем придерживаться (\ref{ris:data_info_chart_1}. ). 

\begin{figure}[h!]
	\begin{center}
		

\tikzset{every picture/.style={line width=0.75pt}} %set default line width to 0.75pt        

\begin{tikzpicture}[x=0.75pt,y=0.75pt,yscale=-1,xscale=1]
%uncomment if require: \path (0,239); %set diagram left start at 0, and has height of 239

%Rounded Rect [id:dp8355723819738923] 
\draw [fill={rgb, 255:red, 184; green, 233; blue, 134 }  ,fill opacity=1 ]  (98,119.5) .. controls (98,115.08) and (101.58,111.5) .. (106,111.5) -- (202.97,111.5) .. controls (207.39,111.5) and (210.97,115.08) .. (210.97,119.5) -- (210.97,143.5) .. controls (210.97,147.92) and (207.39,151.5) .. (202.97,151.5) -- (106,151.5) .. controls (101.58,151.5) and (98,147.92) .. (98,143.5) -- cycle ;
%Rounded Rect [id:dp6680308810096447] 
\draw [fill={rgb, 255:red, 184; green, 233; blue, 134 }  ,fill opacity=1 ]  (270,119.5) .. controls (270,115.08) and (273.58,111.5) .. (278,111.5) -- (374.97,111.5) .. controls (379.39,111.5) and (382.97,115.08) .. (382.97,119.5) -- (382.97,143.5) .. controls (382.97,147.92) and (379.39,151.5) .. (374.97,151.5) -- (278,151.5) .. controls (273.58,151.5) and (270,147.92) .. (270,143.5) -- cycle ;
%Straight Lines [id:da6784338312341602] 
\draw    (211,131) -- (268,131) ;
\draw [shift={(268,131)}, rotate = 180] [color={rgb, 255:red, 0; green, 0; blue, 0 }  ][line width=0.75]    (10.93,-3.29) .. controls (6.95,-1.4) and (3.31,-0.3) .. (0,0) .. controls (3.31,0.3) and (6.95,1.4) .. (10.93,3.29)   ;

% Text Node
\draw (127,125) node [anchor=north west][inner sep=0.75pt]   [align=left] {Данные};
% Text Node
\draw (286,125) node [anchor=north west][inner sep=0.75pt]   [align=left] {Информация};

% Text Node
\draw (90,160) node [anchor=north west][inner sep=0.75pt]   [align=left] {То, что зафиксировано};
% Text Node
\draw (272,160) node [anchor=north west][inner sep=0.75pt]   [align=left] {То, что обработано};


\end{tikzpicture}

		\caption{Связь данных и информации}
		\label{ris:data_info_chart_1}
	\end{center}
\end{figure}

Под данными мы будем понимать совокупность сведений, зафиксированных на определенном носителе в форме пригодной для постоянного хранения обработки и интерпретации. Например замеры давления на приеме УЭЦН записанные в базу данных - это данные. Главное, что они записаны и к ним можно получить доступ при необходимости. 
Под информацией мы будем понимать результат преобразования и анализа данных направленный на практическую деятельность -- на принятие решений. Хотя информация должна обрести некоторую форму представления, то есть превратиться в данные, чтобы ей можно было обмениваться, информация есть в первую очередь интерпретация (смысл) такого представления. Информация всегда привязана к практической деятельности. Без применения, хотя бы потенциального, информация превращается в данные. 

Инженерная деятельность подразумевает постоянные действия для решения поставленных задач. Для нефтяного инжиниринга это эксплуатация месторождений углеводородов -- бурение скважине, проведение геолого-технических мероприятий. Действия или решения требуют необратимой траты ресурсов -- времени, материалов, денег. Из за ограниченности ресурсов возникает необходимость тщательного планирования действий -- принятия решений о необходимости действий. Мы часто будем использовать термин принятие решений и говорить о лице принимающем решение (ЛПР) для обозначения практической деятельности. Принятие решений -- это тот рубеж, который является целью исследований. Для принятия решений строятся модели, собираются данные и генерируется информация. 


Цепочку связи данных, информации и принятия решений можно отобразить схеме (рисунок  \ref{ris:data_info_decision_chart_1}). 

\begin{figure}[h!]
	\begin{center}
		


\tikzset{every picture/.style={line width=0.75pt}} %set default line width to 0.75pt        

\begin{tikzpicture}[x=0.75pt,y=0.75pt,yscale=-1,xscale=1]
%uncomment if require: \path (0,300); %set diagram left start at 0, and has height of 300

%Rounded Rect [id:dp2235462280724534] 
\draw   (135.2,68) .. controls (135.2,63.58) and (138.78,60) .. (143.2,60) -- (227.97,60) .. controls (232.39,60) and (235.97,63.58) .. (235.97,68) -- (235.97,92) .. controls (235.97,96.42) and (232.39,100) .. (227.97,100) -- (143.2,100) .. controls (138.78,100) and (135.2,96.42) .. (135.2,92) -- cycle ;
%Rounded Rect [id:dp17536142021697776] 
\draw   (293,68) .. controls (293,63.58) and (296.58,60) .. (301,60) -- (388.97,60) .. controls (393.39,60) and (396.97,63.58) .. (396.97,68) -- (396.97,92) .. controls (396.97,96.42) and (393.39,100) .. (388.97,100) -- (301,100) .. controls (296.58,100) and (293,96.42) .. (293,92) -- cycle ;
%Straight Lines [id:da11139667469680425] 
\draw    (235,80) -- (292.2,80) ;
\draw [shift={(294.2,80)}, rotate = 180] [color={rgb, 255:red, 0; green, 0; blue, 0 }  ][line width=0.75]    (10.93,-3.29) .. controls (6.95,-1.4) and (3.31,-0.3) .. (0,0) .. controls (3.31,0.3) and (6.95,1.4) .. (10.93,3.29)   ;
%Rounded Rect [id:dp3985813066419075] 
\draw   (457,68) .. controls (457,63.58) and (460.58,60) .. (465,60) -- (571.2,60) .. controls (575.62,60) and (579.2,63.58) .. (579.2,68) -- (579.2,92) .. controls (579.2,96.42) and (575.62,100) .. (571.2,100) -- (465,100) .. controls (460.58,100) and (457,96.42) .. (457,92) -- cycle ;
%Straight Lines [id:da985211792743889] 
\draw    (399,80) -- (456.2,80) ;
\draw [shift={(458.2,80)}, rotate = 180] [color={rgb, 255:red, 0; green, 0; blue, 0 }  ][line width=0.75]    (10.93,-3.29) .. controls (6.95,-1.4) and (3.31,-0.3) .. (0,0) .. controls (3.31,0.3) and (6.95,1.4) .. (10.93,3.29)   ;

% Text Node
\draw (155,73) node [anchor=north west][inner sep=0.75pt]   [align=left] {Данные};
% Text Node
\draw (308,73) node [anchor=north west][inner sep=0.75pt]   [align=left] {Информация};
% Text Node
\draw (460,73) node [anchor=north west][inner sep=0.75pt]   [align=left] {Принятие решений};


\end{tikzpicture}

		\caption{Связь данных, информации и принятия решений}
		\label{ris:data_info_decision_chart_1}
	\end{center}
\end{figure}

Схема показывает, что информация напрямую связана с принятием решений. Если мы не понимаем зачем необходима информация -- на принятие какого решения она работает, то она может оказаться бессмысленной. В приведенной схеме процесс преобразования данных в информацию можно назвать исследованием. 
Цель исследования -- обеспечить процесс принятия решений информаций. И аналогично информации -- исследование имеет смысл только если направлено на принятие решений. 
\marginpar{
	\href{https://ru.wikipedia.org/wiki/исследование}{Исследование} 
	\includegraphics[scale=0.4]{pics/qr_wikipedia_investigation.png} } 
Схему можно еще дополнить добавив в нее инструмент для преобразования данных в информацию и принятия решений на основе информации -- модель. Чаще всего имеются в виду математические модели в виде компьютерных алгоритмов. Именно их мы будет использовать, но на их месте могут быть и другие модели (например интуиция лица принимающего решения или его опыт). 

\begin{figure}[h!]
	\begin{center}
		

\tikzset{every picture/.style={line width=0.75pt}} %set default line width to 0.75pt        

\begin{tikzpicture}[x=0.75pt,y=0.75pt,yscale=-1,xscale=1]
%uncomment if require: \path (0,300); %set diagram left start at 0, and has height of 300

%Rounded Rect [id:dp597190226460943] 
\draw   (113.2,151) .. controls (113.2,146.58) and (116.78,143) .. (121.2,143) -- (208.97,143) .. controls (213.39,143) and (216.97,146.58) .. (216.97,151) -- (216.97,175) .. controls (216.97,179.42) and (213.39,183) .. (208.97,183) -- (121.2,183) .. controls (116.78,183) and (113.2,179.42) .. (113.2,175) -- cycle ;
%Rounded Rect [id:dp3409304527040422] 
\draw   (274,151) .. controls (274,146.58) and (277.58,143) .. (282,143) -- (378.97,143) .. controls (383.39,143) and (386.97,146.58) .. (386.97,151) -- (386.97,175) .. controls (386.97,179.42) and (383.39,183) .. (378.97,183) -- (282,183) .. controls (277.58,183) and (274,179.42) .. (274,175) -- cycle ;
%Straight Lines [id:da9955922595564108] 
\draw    (216,163) -- (273.2,163) ;
\draw [shift={(275.2,163)}, rotate = 180] [color={rgb, 255:red, 0; green, 0; blue, 0 }  ][line width=0.75]    (10.93,-3.29) .. controls (6.95,-1.4) and (3.31,-0.3) .. (0,0) .. controls (3.31,0.3) and (6.95,1.4) .. (10.93,3.29)   ;
%Rounded Rect [id:dp45297167935805027] 
\draw   (451,151) .. controls (451,146.58) and (454.58,143) .. (459,143) -- (574.2,143) .. controls (578.62,143) and (582.2,146.58) .. (582.2,151) -- (582.2,175) .. controls (582.2,179.42) and (578.62,183) .. (574.2,183) -- (459,183) .. controls (454.58,183) and (451,179.42) .. (451,175) -- cycle ;
%Straight Lines [id:da11147374736012572] 
\draw    (388.2,161.94) -- (447.2,162) ;
\draw [shift={(449.2,162)}, rotate = 180.06] [color={rgb, 255:red, 0; green, 0; blue, 0 }  ][line width=0.75]    (10.93,-3.29) .. controls (6.95,-1.4) and (3.31,-0.3) .. (0,0) .. controls (3.31,0.3) and (6.95,1.4) .. (10.93,3.29)   ;
%Rounded Rect [id:dp851243700176072] 
\draw   (211.2,209) .. controls (211.2,204.58) and (214.78,201) .. (219.2,201) -- (442.2,201) .. controls (446.62,201) and (450.2,204.58) .. (450.2,209) -- (450.2,233) .. controls (450.2,237.42) and (446.62,241) .. (442.2,241) -- (219.2,241) .. controls (214.78,241) and (211.2,237.42) .. (211.2,233) -- cycle ;
%Straight Lines [id:da1477687213170571] 
\draw    (245,200.5) -- (245.57,165) ;
\draw [shift={(245.6,163)}, rotate = 450.92] [color={rgb, 255:red, 0; green, 0; blue, 0 }  ][line width=0.75]    (10.93,-3.29) .. controls (6.95,-1.4) and (3.31,-0.3) .. (0,0) .. controls (3.31,0.3) and (6.95,1.4) .. (10.93,3.29)   ;
%Straight Lines [id:da9162736044666429] 
\draw    (419,199.5) -- (419.57,164) ;
\draw [shift={(419.6,162)}, rotate = 450.92] [color={rgb, 255:red, 0; green, 0; blue, 0 }  ][line width=0.75]    (10.93,-3.29) .. controls (6.95,-1.4) and (3.31,-0.3) .. (0,0) .. controls (3.31,0.3) and (6.95,1.4) .. (10.93,3.29)   ;

% Text Node
\draw (137,153.5) node [anchor=north west][inner sep=0.75pt]   [align=left] {Данные};
% Text Node
\draw (285,154.5) node [anchor=north west][inner sep=0.75pt]   [align=left] {Информация};
% Text Node
\draw (460,153.5) node [anchor=north west][inner sep=0.75pt]   [align=left] {Принятие решений};
% Text Node
\draw (304,212.5) node [anchor=north west][inner sep=0.75pt]   [align=left] {Модель};


\end{tikzpicture}

		\caption{Связь данных, информации, принятия решений и моделирования}
		\label{ris:data_model_chart_1}
	\end{center}
\end{figure}

Именно такого подхода мы будем придерживаться в курсе и данном пособии. Такое определение, хотя и несколько отличается от классических, обладает рядом преимуществ. Например позволяет оценить когда надо проводить исследования и сколько ресурсов на них можно потратить или увидеть как можно некоторые виды деятельности, изначально не направленные на получение информации, рассматривать в качестве исследований и извлечь из них дополнительную ценность.
 \marginpar{
 	\href{https://www.gazprom-neft.ru/press-center/sibneft-online/archive/2018-may/1589542/} {Газпром нефть. Цифровизация — это фундаментальный тренд}.
 	\includegraphics[scale=0.4]{pics/qr_digitalization.png} }  
Например с развитием информационных технологий в нефтедобычи все больше информации о нормальной эксплуатации скважины накапливается в базах данных добывающих компаний. Эти данные содержат информацию о том как работают скважины и месторождения. Если их рассмотреть как исследования -- их можно преобразовать в информацию и использовать ее для принятия решений, например о проведении ГТМ, без необходимости траты ресурсов на проведение классических исследований. Это направление сейчас активно развивается в нефтяной промышленности под знамёнами цифровизации и в рамках курса будет неоднократно обсуждаться.

Подумайте как можно на приведённые схемы добавить категорию - знание? Как могут быть связаны исследования и знания?

\section{Ожидаемая ценность в условиях неопределённости -- EMV}

Исследования почти всегда идут рядом с неопределённостями различного вида. Неопределённости могут быть связаны с параметрами изучаемых объектов. Например планируя пробурить скважину, мы хотели бы знать ее дебит. Мы знаем, что дебит будет зависеть от проницаемости пласта, но проницаемость мы не знаем -- мы находимся в ситуации неопределённости. 

Мы  можем описать неопределённость проницаемости задав вероятные исходы для значения проницаемости. Пусть мы ожидаем, что величина проницаемости задаётся следующим распределением:
\begin{itemize}
	\item $k_1 = 10$ мД с вероятность $P_1 = 0.2$; 
	\item $k_2 = 20$ мД с вероятность $P_2 = 0.6$; 
	\item $k_3 = 30$ мД с вероятность $P_3 = 0.2$. 
\end{itemize}

тогда дебиты скважины могут определены из выражения

\begin{equation}
	Q = \frac{kh}{18.41 \mu B} \frac{\Delta P}{  \left( ln\dfrac{r_e}{r_w} + S\right) }
	\label{eq:eq_q}
\end{equation}


а доход от эксплуатации скважины в течении года 

\begin{equation}
	MV = Q \cdot  \Delta T \cdot  Price
	\label{eq:eq_MV}
\end{equation}

где 

$MV$ -- доход от эксплуатации (monetary value);

$Q$ -- дебит скважины ожидаемый;
 
$\Delta T$ -- период времени за который проходит оценка;

$Price$ -- цена единицы добычи нефти.

Тогда подставляя \eqref{eq:eq_q} в \eqref{eq:eq_MV} получим 

\begin{equation}
	MV = \alpha k
	\label{eq:eq_MV_2}
\end{equation}

где 

$$\alpha =  \Delta T \cdot  Price \cdot \frac{h}{18.41 \mu B} \frac{\Delta P}{  \left( ln\dfrac{r_e}{r_w} + S\right)}
$$ 

параметр, который мы предполагаем постоянным в рамках данной модели. 

Тогда ожидаемую доходность с учетом неопределенности можно получить как математическое ожидание доходности с учетом заданного распределения. 

\begin{equation}
	EMV = \sum_{i} MV_i P_i
	\label{eq:eq_EMV_1}
\end{equation}

Величину ожидаемой доходности можно представить в виде графической диаграммы, приведенной на рисунке \ref{ris:chance_node}.

%\begin{figure}
%	
%	\begin{tikzpicture}[declare function={
%			f(\x,\y)=(\y/\x)* (1/sqrt(2*pi*1))*exp(-(ln(\x/\y)-0)^2/(2*1);
%			g(\x,\y)=(1/sqrt(2*pi*1))*exp(-(\x-\y)^2/(2*1);
%		}]
%		\begin{axis}[every axis plot post/.append style={
%				mark=none,domain=0.01:200,samples=50,smooth},		
%			axis x line*=bottom, % no box around the plot, only x and y axis
%			axis y line*=left, % the * suppresses the arrow tips
%			enlargelimits=upper] % extend the axes a bit to the right and top
%			\addplot[color=green!60!black,
%			very thick]
%			{f(x,100)};	
%			
%		\end{axis}
%	\end{tikzpicture} 
%
%\end{figure}


\begin{figure}[h!]
	\tikzset{every picture/.style={line width=0.75pt}} %set default line width to 0.75pt        
	\centering
	\begin{tikzpicture}[x=0.75pt,y=0.75pt,yscale=-1,xscale=1]
		%uncomment if require: \path (0,300); %set diagram left start at 0, and has height of 300
		
		%Straight Lines [id:da9914598188026227] 
		\draw    (125,128) -- (237.4,41) ;
		%Straight Lines [id:da8494600211227674] 
		\draw    (125,128) -- (237.4,215.45) ;
		%Straight Lines [id:da3455826146339449] 
		\draw    (125,128) -- (237.4,128) ;
		%Shape: Circle [id:dp4694470217534936] 
		\draw  [fill={rgb, 255:red, 248; green, 231; blue, 28 }  ,fill opacity=1 ] (103.95,128) .. controls (103.95,116.37) and (113.37,106.95) .. (125,106.95) .. controls (136.63,106.95) and (146.05,116.37) .. (146.05,128) .. controls (146.05,139.63) and (136.63,149.05) .. (125,149.05) .. controls (113.37,149.05) and (103.95,139.63) .. (103.95,128) -- cycle ;
		%Shape: Triangle [id:dp6993619331790251] 
		\draw  [fill={rgb, 255:red, 184; green, 233; blue, 134 }  ,fill opacity=1 ] (237.4,41) -- (267.48,21.01) -- (267.39,61.11) -- cycle ;
		%Shape: Triangle [id:dp5948591148628843] 
		\draw  [fill={rgb, 255:red, 184; green, 233; blue, 134 }  ,fill opacity=1 ] (237.4,127.97) -- (267.48,107.98) -- (267.39,148.08) -- cycle ;
		%Shape: Triangle [id:dp600205010959115] 
		\draw  [fill={rgb, 255:red, 184; green, 233; blue, 134 }  ,fill opacity=1 ] (237.4,215.42) -- (267.48,195.43) -- (267.39,235.53) -- cycle ;
		
		% Text Node
		\draw (174,50) node [anchor=north west][inner sep=0.75pt]    {$p_{1}, k_1$};
		% Text Node
		\draw (174,106) node [anchor=north west][inner sep=0.75pt]    {$p_{2}, k_2$};
		% Text Node
		\draw (174,146) node [anchor=north west][inner sep=0.75pt]    {$p_{3}, k_3$};
		% Text Node
		\draw (277,19) node [anchor=north west][inner sep=0.75pt]    {$MV_{1}$};
		% Text Node
		\draw (277,106) node [anchor=north west][inner sep=0.75pt]    {$MV_{2}$};
		% Text Node
		\draw (277,193) node [anchor=north west][inner sep=0.75pt]    {$MV_{3}$};
		% Text Node
		\draw (92,162) node [anchor=north west][inner sep=0.75pt]    {$\sum p_{i} MV_{i}$};
	\end{tikzpicture}
	\label{ris:chance_node}
	\caption{Графическое представление для вычисления ожидаемой доходности EMV}
\end{figure}

Здесь мы, следуя соглашения для отрисовки деревьев решений, в виде круга обозначили вероятностный узел -  выходами из которого являются ветви дерева, представляющие возможные дискретные исходы задаваемым природой или постановкой задачи - исходы на которые мы не можем повлиять. Конечные исходы обозначаем треугольниками. Каждая ветвь выходящая из вероятностного узла характеризуется вероятностью и значением проницаемости (для нашей задачи), каждый исход описывается величиной доходности при заданной вероятности. 


\section{Ценность информации полученной в ходе исследований -- VOI}

Хорошим вопросом касательно любых исследований и даже шире -- любых расчётов и инженерных моделей -- является вопрос -- зачем всем этим надо заниматься? Задав этот вопрос, можно услышать много ответов. Часто говорят -- чтобы лучше изучить месторождение, чтобы определить параметры месторождения или скважины, чтобы рассчитать оптимальный режим работы скважины. Такие ответы верны, но в тоже время слишком расплывчаты. Попробуйте поставить себя на место главного инженера добывающего предприятия и оценить -- выделили бы вы ценные ресурсы молодому специалисту который предлагает лучше изучить какую-то скважину? 

Лица принимающие решения (главный инженер добывающего предприятия -- хорошая модель ЛПР для нашего курса) любят более конкретные обоснования как правило. Лучше всего экономические.

Один из ответов на вопрос -- "Зачем проводить исследования?" можно получить с использованием понятия ценности информации (Value of Information). 

\marginpar{
        \href{https://ru.wikipedia.org/wiki/ценность_информации}{Ценность информации} 
        \includegraphics[scale=0.4]{pics/qr_wikipedia_VOI.png} 
        }

Согласно упрощённой версии этого подхода можно допустить, что ценность информации полученного в ходе исследования равна разнице в эффективности принятого решения с учётом этой информации и без ее учета. 
$$VOI = EMV_{inf}-EMV_{0}$$

Здесь $VOI$ - ценность информации, $EMV_{inf}$ - ожидаемая ценность принятого решения с учетом информации (estimated monetary value), $EMV_{0}$ - ожидаемая ценность принятого решения без учета информации.

То есть ценность исследования, а следовательно и целесообразность его проведения (затрат ресурсов на его проведение) целиком зависит от эффективности принятых решений. 

Такой подход является явным упрощением. По приведённым ссылкам можно найти и другие подходы, где в частности предлагается учесть и менее явные последствия. Но его преимущество как раз в простоте. Он позволяет количественно оценить эффект от проведения исследований, соотнести его с затратами на проведение исследований и построить критерий принятия решений, который уже можно далее обсуждать и улучшать. 


Существует много подходов к построению критерия принятия решений о необходимости проведения исследований [ссылка на making good decisions]. Мы рассмотрим один из них, широко применяемый в различных областях деятельности, особенной в экономике и компьютерных науках - метод построения деревьев решений. \cite{AL_appl_patt_2007}

\section{Деревья решений для планирования исследований}

% translation from https://github.com/SilverDecisions/SilverDecisions/wiki/1.-Decision-tree-model
Последовательность и неопределенность присущи практическому принятию решений. Первое означает, что лица, принимающие решения, должны рассматривать многоступенчатые стратегии, охватывающие несколько действий, следующих друг за другом, а не только одно действие. Вторая означает, что отдача, получаемая лицами, принимающими решения, зависит не только от действий, но и от внешних событий ( состояний мира), которые часто могут быть восприняты как случайные. Действия и реакции обычно переплетаются, что еще больше усложняет картину. Деревья решений используются в качестве модели, помогающей обнаружить, понять и передать структуру таких проблем, связанных с принятием решений.

Ниже представлено простое дерево решений, созданное с помощью программы SilverDecisions (файл SilverDecisions, содержащий это дерево, можно запустить \href{http://silverdecisions.pl/SilverDecisions.html?LOAD_SD_TREE_JSON=https://raw.githubusercontent.com/gubkin-rienm/isp/master/data/decision_tree/simple_invest_decision.json}{здесь}).
\marginpar{
        ссылку там есть - надо сделать QR на нее
        }
%\urldef{\myurl}{http://silverdecisions.pl/SilverDecisions.html?LOAD_SD_TREE_JSON=https://raw.githubusercontent.com/gubkin-rienm/isp/master/data/decision_tree/simple_invest_decision.json}

\begin{figure}[h!]
	\includegraphics[width= 15cm]{pics/simple_decision_tree_1.png} 
	\label{fig:sample}
	\caption{Простое дерево решений}
\end{figure}

Модель дерева решений описывает и визуализирует последовательность принятия решения в условиях неопределенности на древовидной диаграмме. Это означает, что деревья решений могут быть полезны в таких задачах:
\begin{itemize}
    \item лицо, принимающее решение, выполняет несколько действий, следуя друг за другом,
    \item состояния мира могут различаться в зависимости от уже принятых решений,
    \item некоторые решения могут привести к более точным оценкам вероятности этих состояний.
\end{itemize}

На древовидной диаграмме представлены возможные решения, которые необходимо принять, независимые события, которые могут произойти, и результаты, связанные с комбинациями этих решений и событий. Необходимо определить два параметра: вероятности событий и их значения или стоимости. Первый параметр представляет вероятность получения определенного состояния мира. Поскольку возможные состояния мира в рамках одной реакции на самом деле являются конкурирующими событиями, сумма их вероятностей должна быть равна 1. 
Тогда значения или стоимости, выраженные в некоторой шкале, например в деньгах, означают платежи (изменение стоимости) как следствие решения или состояния мира. 
Это может быть как прибыль, так и убытки. Модель дерева решений включает еще одно понятие: EMV - ожидаемая значение или ожидаемая ценность (или ожидаемая полезность). Она вычисляется как вероятностно-взвешенное среднее значений для конкурирующих наборов решений и событий. Ожидаемое значение EMV показывает, сколько человек может заработать или потерять, принимая оптимальные решения (это означает такие решения, которые максимизируют прибыль и минимизируют убытки). Наконец, результат, связанный с решениями и событиями, представляет собой общее последствие набора решений и событий во всем процессе принятия решения. Он может быть истолкован как отдача лица, принимающего решение - результат как его решений, так и произошедших независимых событий.

Деревья решений с их простой для понимания структурой являются отличным инструментом для решения задач анализа. Они позволяют исследовать возможные результаты принятия решений и помогают выбирать между различными направлениями действий. Основной целью модели дерева решений является определение наилучшей возможной политики, которая представляет собой наибольшую отдачу или наименьший убыток.


Дерево решений строится по направленному графику слева направо, с набором узлов, которые разбиваются на три разрозненных множества:
\begin{itemize}
    \item узлы решения - типично представленные в виде квадратов,
    \item случайные узлы, представленные в виде кругов,
    \item терминальные узлы представлены в виде треугольников.
\end{itemize}

Крайняя левая вершина называется корневой вершиной и является первой вершиной принятия решения (первый красный квадрат слева - см. выше простую модель принятия инвестиционного решения - рисунок \ref{fig:sample} ). В узлах принятия решений выбирает именно тот, кто принимает решение, т.е. выбирает ровно одну из ветвей, выходящих из этого узла. Эти ветви представляют собой набор доступных альтернатив решения (действий). В случайном узле (желтые кружки - дерево образцов выше) каждая из вытекающих из него ребер - реакция - выбирается случайным образом с заданной вероятностью события. Терминальные узлы (синие треугольники на дереве выборки выше) представляют собой результат последовательности действий/реакций от корневого узла к данному конкретному терминальному узлу. Терминальный узел является конечной точкой: никакие решения не могут быть приняты, и никакие события не могут произойти после этого.

В приложении SilverDecisions вероятность событий и значения, связанные с этими событиями или решениями, определяются по краям. Ожидаемые значения, рассчитанные для каждого набора решений/событий, отображаются в каждом узле решения/шанса, а терминальные узлы показывают результаты и вероятности того, что событие окажется в указанном терминальном узле.

Обратите внимание, что каждое ребро совмещается с двумя узлами: левый, из которого выходит ребро, называется родительским узлом, а второй, находящийся справа, называется дочерним узлом. Поддерево - это еще один термин, связанный с деревьями решений - оно представляет собой ту часть дерева, которая начинается в любом дочернем узле, и каждый из них вместе с любыми потомками образует поддерево. Например, поддерево, уставившееся в корневой узел, представляет собой целое дерево.






\section{Задачи с примерами решения}

Деревья решений являются удобным инструментом для формализации процесса принятия решений. Они достаточно широко используются в нефтяной индустрии [привести ссылки на книгу и статьи]. 

Для построения деревьев решений можно использовать ручку, бумагу и калькулятор, так расчеты достаточно просты. Но удобнее использовать специализированные программные инструменты. Далее мы будем использовать сайт \url{http://silverdecisions.pl/} предоставляющий удобный, и что не менее важно, бесплатный и открытый программный комплекс для построения и анализа деревьев решений.

\subsection{Исследования на разведочной скважине}
    
    
Оцените стоит ли проводить комплекс исследований на разведочной скважине со следующими параметрами:
\begin{itemize}
    \item Стоимость 1 млн руб
    \item Ожидаемая информация – фильтрационные параметры пласта, уточнение строение пласта
    \item Вероятность успешности исследования (получения какой то информации) 70%
    \item Ожидается что исследование подтвердит увеличение запасов на 15% (60% что запасы увеличатся)
    \item Текущие извлекаемые запасы – 1 млн т. нефти, стоимость 1 т.нефти в запасах – 1 тыс. руб.
\end{itemize}     
Стоит ли проводить исследование? 

    
\subsection{КВД на добывающей скважине}

Оцените стоит ли проводить КВД на эксплуатационной скважине с дебитом 50 т/нефти, Рзаб = 50 атм, Рпл = 250 атм
\begin{itemize}
    \item Стоимость исследования 1 млн руб
    \item Длительность 1 неделя (скважина остановлена)
    \item Ожидаемая информация – фильтрационные параметры пласта, скин фактор
    \item Вероятность успешности исследования (получения какой то информации) 70%
    \item Ожидается что исследование подтвердит наличие положительного скин фактора
    \begin{itemize}
        \item S=0 с вероятностью 50%
        \item S=5 с вероятностью 40%
        \item S=15 с вероятностью 10%
    \end{itemize}
    \item Стоимость кислотной обработки снижающей скин в 2 раза 1 млн руб. Длительность эффекта 1 год
    \item Стоимость нефти 1 т = 10 тыс. руб.
\end{itemize}
Стоит ли проводить исследование? 

\subsection{Исследование фонтанирующей скважины}
Имеется фонтанирующая скважина. Дебит нефти 30 т/сут. Воды нет. Имеется водоносный пласт под продуктивным. Перемычка 10 м. Вероятность получения прорыва воды после ГРП 50 \%. При прорыве воды рост обводненности до 90 \% без увеличения дебита нефти. Без прорыва воды увеличение продуктивности в 3 раза. Стоимость ГРП 1 млн. руб.

Оценить что делать на скважине?

Ничего не делать
Сделать ГРП
Провести исследование и по его результатам ГРП
Стоимость исследования 100 т.р.

Вероятность выявления скважины где ГРП будет успешен 60%

\chapter{Промысловые исследования}

\section{Исследования физико-химических свойств пластовых флюидов}

Здесь основные определения и теоретический минимум для последующих разделов. 
И набор мелких "коварных" задач для контроля знаний.
\subsection{Теория и определения}
\begin{itemize}
    \item Пластовые флюиды. Вода, нефть, газ.
    \item Модели флюидов. Модель нелетучей нефти. Композиционная модель. Основные предположения и отличия
    \item Параметры нефти для модели нелетучей нефти. Из зависимость от давления и температуры. 
    \item Параметры газа. z фактор. Коэффициент Джоуля Томсона для углеводородных газов. 
    \item Параметры воды и пара.
    
\end{itemize}
\subsection{Методы определения свойств флюидов}
Отбор глубинных проб, замеры на поверхности. Экспресс методы исследования.
\subsection{Задачи}
Расчеты с использованием Excel, Python, специализированное ПО
\begin{itemize}
    \item Зависимости параметров нефти воды и растворенного газа от давления и температуры (можно с использованием унифлок Excel)
    \item Построить график плотности воды при одновременно изменяющихся давлении и температуре
    \item Построить график зависимости доли свободного газа  от давления и температуры в потоке
    \item Оценить изменение температуры газа и ГЖС при изменении давления из за эффекта Джоуля Томсона
    \item Рассчитать изменение свойств воды и пара при нагреве.
    \item оценка газового фактора по данным замера расхода газа и нефти - учет растворенного в нефти газа на замерной с высоким давлением.

    % недавно было обнаружено что график достаточно странно выглядит :) 
    %\item Расчеты закона Дарси для трубы и простые задачи по линейному потоку в пласте. Определить перепад давления в трубе, оценить скорость движения жидкости и так далее.
\end{itemize}
    

\section{Теория}
\begin{itemize}
    \item Связь давления и движения флюидов в трубах и скважине. Формула Дарси-Вейсбах, уравнение Навье Стокса. Важность учета PVT при моделировании потоков
    \item Методы расчета распределения давления в потоке. Краткой обзор методов и программ
    \item Принципы измерения давления, температурная компенсация, Приборы для измерения давления в эксплуатируемой скважине
    \item Забойное давление - ключевой параметр для контроля работы скважины. Методы контроля забойного давления - прямые измерения и косвенные оценки для различных типов скважин. Частота измерения забойного давления. Цели и задачи измерения забойного давления.
    \item хранение результатов измерений давления. Шахматка, Техрежим, БСИ ЭЦН, корпоративные базы данных.
\end{itemize}

\section{Задачи}

Задачи с использованием Excel и пакетов программ. 

\begin{itemize}
    \item Расчет распределения давления в скважине с использованием гидравлических корреляций. Построение кривой распределения давления в скважине.
    \item Расчет забойного давления по динамическому уровню в скважине с ЭЦН. Анализ отжима динамического уровня. 
\end{itemize}


\section{Теория}
\begin{itemize}
    \item Связь давления и движения флюидов в трубах и скважине. Формула Дарси-Вейсбах, уравнение Навье Стокса. Важность учета PVT при моделировании потоков
    \item Методы расчета распределения давления в потоке. Краткой обзор методов и программ
    \item Принципы измерения давления, температурная компенсация, Приборы для измерения давления в эксплуатируемой скважине
    \item Забойное давление - ключевой параметр для контроля работы скважины. Методы контроля забойного давления - прямые измерения и косвенные оценки для различных типов скважин. Частота измерения забойного давления. Цели и задачи измерения забойного давления.
    \item хранение результатов измерений давления. Шахматка, Техрежим, БСИ ЭЦН, корпоративные базы данных.
\end{itemize}

\section{Задачи}

Задачи с использованием Excel и пакетов программ. 

\section{Задачи}
\begin{itemize}
    \item Расчет распределения температуры в скважине с учетом притока разных флюидов в скважине
    \item Калориметрическое смешивание
    
\end{itemize}


\section{Дебитометрия в скважине и на поверхности. Принципы замера дебита и виртуальный расходомер.}


\subsection{Теория}
\begin{itemize}
    \item Принципы измерения дебита (расхода). Оборудования для измерения дебита на поверхности и в скважине
    \item Замер обводненности и газового фактора продукции
    \item Замеры количества механических примесей
    \item Профили притока к скважине. Замер профиля притока с использованием скважинного оборудования
\end{itemize}

\subsection{Задачи}

\begin{itemize}
    \item Оценка дебита по закону Бернулли - на гидравлическом сужении. Оценка дебита по характеристике потока через штуцер
    \item Оценка обводненности по данным замером плотности флюидов и плотности смеси
    \item Оценка газового фактора по данным эксплуатации скважины - перепад давления в скважине и на штуцере
    \item Оценка газового фактора по данным отжима динамического уровня
    \item Расчет профиля изменения давления и профиля притока в горизонтальной скважине (учет по плотности притока и диаметру ствола скважины)
    
\end{itemize}
%\section{Выявление нарушений герметичности ствола скважины}
    

\chapter{ГДИС. Простые решения}

%\section{Гидродинамические исследования скважин и пластов}


В основе гидродинамических исследований скважин и пластов, как и для других видов исследований, лежит сопоставление результатов измерений на скважинах с математическими моделями их работы. Используются различные виды измерений - дебиты по фазам, давления на устье и забое, динамика изменения параметров во времени. Если найти такую модель, которая объясняла бы наблюдаемые изменения параметров, тогда можно надеяться, что она сможет дать хороший прогноз.

Необходимость сопоставления измерений с моделями требует детального понимания того как устроены модели. Месторождение, продуктивный пласт, скважины - являются сложными объектами работу которых определяет множество факторов и физических процессов. Построение единой детальной модели, способной учесть всю доступную информацию - на практике оказывается чрезвычайно сложной задачей. Попытка построить такую модель неизбежно приводит к тому, что для учета всей имеющейся информации в модель приходится вносить большой объем данных и еще больше догадок и предположений (не упоминая даже того, что и в имеющейся информации мы часто не очень то уверены). Это приводит к тому, что прогноз (и решения зависящие от него) могут оказаться неверными. 

Вместо одной супер модели на практике более эффективно применять иерархию моделей. Для каждой задачи, можно подобрать отдельную, относительно простую модель позволяющую принять необходимое решение. Хорошо иллюстрирует подход к выбору модели афоризм -- модель должна быть максимально простой - но не проще чем необходимо. 

Применяя иерархию моделей, важным становится понимать как устроены разные модели, начиная от самых простых - и далее с различными усложнениями, и дополнительными эффектами.

Этот раздел посвящен разбору простых моделей фильтрации, которые лежат в основе методов интерпретации гидродинамических исследований. Также рассматриваются методы проведения практических вычислений по разобранным моделям с использованием компьютерных алгоритмов реализованных в Excel VBA.

Проведение практических расчетов показано с использованием надстройки Unifloc VBA \url{https://github.com/unifloc/unifloc_vba}.

\marginpar{
	\href{https://en.wikipedia.org/wiki/Nondimensionalization}{Unifloc VBA} 
	\includegraphics[scale=0.4]{pics/qr_unifloc.eps} 
}
%\section{Простые решения уравнения фильтрации}

\section{Уравнение фильтрации}

Уравнение фильтрации многофазного потока в пористой среде - основная математическая модель пласта используемая в инженерных приложениях. Используется не само уравнение, а разнообразные частные решения, описывающие те или иные ситуации. Наиболее широко известным примером решения уравнения фильтрации, возможно является, численное решение которое строится трехмерными гидродинамическими симуляторами (eclipse, tnavigator и так далее)

%[[Вывод уравнения фильтрации]]

Как правило, при выводе уравнения фильтрации используются следующие соотношения:

закон Дарси:  

\begin{equation} \label{eq:darcy_law_1}
 u_r=-\frac{k}{\mu}\frac{dp}{dr} 
\end{equation}

уравнение неразрывности: 
\begin{equation} \label{eq:mass_balance_1} 
\frac{1}{r}\frac{\partial\left(r\rho u_r\right)}{\partial r}=-\varphi\frac{\partial p}{\partial t} 
\end{equation}

уравнение состояния: 
\begin{equation} \label{eq:eos_1} 
c_0=\frac{1}{\rho}\frac{\partial\rho}{\partial p} 
\end{equation}

Опираясь на эти соотношения уравнение фильтрации в радиальной форме можно привести к виду.

\begin{equation} \label{eq:diff_eq_1} 
\frac{\partial ^2 p }{\partial r^2} + \frac{1}{r} \frac{\partial p}{\partial r} = \frac{\varphi \mu c_t}{k} \frac{\partial p}{\partial t} 
\end{equation}

Здесь используются следующие обозначения

$u_r$ - скорость фильтрации в направлении $r$, м/сек

$k$ - проницаемость, м$^2$

$\mu$ - вязкость флюида, Па с

$p$ - давление, Па 

$r$ - расстояние, м 

$\rho$ - плотность флюида, кг/м$^3$

$\varphi$ - пористость породы, доли единиц.

$c_t$ - общая сжимаемость породы и флюида, 1/Па

\subsection{Вывод уравнения фильтрации}

\input{chapters/welltest_basics_diff_eq_derivation}

\section{Стационарные решения уравнения фильтрации}

\input{chapters/welltest_basics_steady_state_solutions}

\section{Нестационарные решения уравнения фильтрации}

Для установившегося режима фильтрации давление в пласте не меняется. Для псевдо-установившегося режима постоянным остается перепад давления между пластом и забоем. После запуска, остановки или изменения режима работы скважины эти условия не выполняются. Давление в различных точках пласта может меняться по разному. Такой режим называют неустановившимся, а решения его описывающие нестационарными (зависят от времени).

\subsubsection{Безразмерные переменные}

Часто для анализа уравнений неустановившейся фильтрации используются безразмерные переменные 

\marginpar{
	\href{https://en.wikipedia.org/wiki/Nondimensionalization}{Обезразмеривание на en.wikipedia.org} 
	\includegraphics[scale=0.4]{pics/qr_Nondimensionalization.eps} 
}

$$ r_D = \frac{r}{r_w} $$
$$ t_D = \frac{kt}{\phi \mu c_t r_w^2}$$
$$ p_D = \frac{2 \pi kh}{q_s B \mu} \left( p_i - p_{wf} \right) $$

Здесь использование единицы измерения СИ. 

$q_s$ - дебит скважины на поверхности, приведенный к нормальным условиям м3/с

$\phi$ - пористость, доли единиц

$\mu$ - вязкость нефти в пласте, Па с

$B$ - объемный коэффициент нефти, м3/м3

$p_i$ - начальное давление в пласте, Па

$p_{wf}$ - давление забойное, Па

$c_t$ - общая сжимаемость системы в пласте, 1/Па

Использование безразмерных переменных позволяет упростить уравнение фильтрации, которое примет вид

$$ \frac{\partial p_D}{ \partial t_D} = \frac{1}{r_D} \frac{ \partial{ \left( r_D \dfrac{\partial p_D}{ \partial r_D} \right) } }{ \partial{r_D} } $$

Решение этого уравнения - функция безразмерного давления от безразмерных времени и расстояния $p_D(r_D, t_D) $

Для практических расчетов удобнее бывает использовать безразмерные переменные полученные для практических метрических единиц измерения. 
$$ r_D = \frac{r}{r_w} $$
$$ t_D = \frac{0.00036 kt}{\phi \mu c_t r_w^2}$$
$$ p_D = \frac{kh}{ 18.41 q_s B \mu} \left( p_i - p_{wf} \right) $$

Здесь использование единицы измерения СИ. 

$q_s$ - дебит скважины на поверхности, приведенный к нормальным условиям м3/сут

$\phi$ - пористость, доли единиц

$\mu$ - вязкость нефти в пласте, сП

$B$ - объемный коэффициент нефти, м3/м3

$p_i$ - начальное давление в пласте, атм

$p_{wf}$ - давление забойное, атм

$c_t$ - общая сжимаемость системы в пласте, 1/атм

дополнительно можно ввести безразмерный коэффициент влияния ствола скважины
$$ C_D = \frac{0.159}{ h \phi \mu c_t r_w^2 } C_s $$

\subsubsection{Решение линейного стока}

Для решения уравнения фильтрации - линейного дифференциального уравнения в частных производных второго порядка необходимо задать начальные и граничные условия. 

Самое простое решение можно получить для случая вертикальной скважины бесконечно малого радиуса запускающейся с постоянным дебитом. Условия соответствующие этому случаю можно выразить следующим образом:

\begin{itemize}
	\item Начальное условие. До запуска скважины в момент времени  $t_D = 0$ давление в пласте равно начальному во всех точках $p=p_i$
	$$ t_D < 0, p_D = 0 $$ 
	\item Граничное условие на скважине.  Условие постоянства дебита на скважине можно трансформировать в граничное условие опираясь на закон Дарси.
	$$ \lim_{r_D \to 0} {r_D \frac{\partial p_D}{\partial r_D}} = -1$$
	\item Граничное условие на бесконечном расстоянии от скважины. Давление в пласте на бесконечно большом расстоянии от скважины равно начальному.
	$$ r_D = \infty, p_D = 0$$
\end{itemize}

\begin{wrapfigure}{r}{0.5\textwidth}
	%\begin{figure}[h!]
	\begin{center}
		\begin{tikzpicture}
			\begin{axis}
				[axis lines = left,
				 width = 0.48\textwidth,
				%xlabel=$x$,
				%ylabel={$Ei_1(x)$},
				]
				\addplot gnuplot[no markers, samples=100, domain = 0:10]{expint(1,x)};
				\addlegendentry{$Ei_1(x)$}
			\end{axis}
		\end{tikzpicture}
		\caption{График функции интегральной экспоненты $Ei_1(x)$.}
		\label{ris:ei1}
	\end{center}
	%\end{figure}
\end{wrapfigure}



В этом случае решение может быть выражено через функцию интегральной экспоненты 
$$ p_D(r_D,t_D) = - \frac{1}{2} Ei \left(- \dfrac{ r_D^2}{4t_d} \right)$$

где $-Ei(-x)$ - интегральная показательная функция.

$$Ei(x)=-\int\limits_{x}^{\infty}\frac{e^{-t}}{t}\,\mathrm dt$$

\marginpar{
	\href{https://www.wolframalpha.com/input/?i=Ei\%28x\%29}{$Ei(x)$ на Wolfram Alpha} 
	\includegraphics[scale=0.4]{pics/qr_ei_wolfram.eps} 
	}

Часто для проведения расчетов, особенно с использованием компьютерных библиотеке расчетов, бывает удобнее пользоваться модифицированной интегральной показательной функцией $Ei_1(x)$ или $E_1(x)$ или $Ei_n(x)$ при $n=1$.
 $$Ei_n(x) = \int\limits_{1}^{\infty}\frac{e^{-tx}}{t^n}\,\mathrm dt $$

График интегральной показательной функции $Ei_1(x)$ приведен на рисунке \ref{ris:ei1}.
Для вещественных положительных $x\in\mathbb R, x>0$ верно $E_1(x) = - Ei( -x)$

Функцию интегральной экспоненты можно представить в виде ряда. 

$$Ei(x)=-\int\limits_{x}^{\infty}\frac{e^{-t}}{t}\,\mathrm dt=\gamma+\operatorname{ln}|-x|+\sum\limits_{n\ge1}\frac{{-x}^n}{n!\cdot n}, \;  x\in\mathbb R,\;$$

\begin{wrapfigure}{r}{0.5\textwidth}
	%\begin{figure}[h!]
	\begin{center}
		\begin{tikzpicture}
			\begin{axis}
				[axis lines = left,
				width = 0.48\textwidth,
				%xlabel=$x$,
				%ylabel={$f(x)$},
				]
				\addplot gnuplot[no markers, samples=100, domain = 0:2]{expint(1,x)};
				\addlegendentry{$Ei_1(x)$}
				\addplot gnuplot[no markers, samples=100, domain = 0:2]{-log(x)-0.5772};
				\addlegendentry{$ln(x)$}
			\end{axis}
		\end{tikzpicture}
		\caption{Сравнение функций интегральной экспоненты $E_1(x)$ и $ln(x)$.}
		\label{ris:ei2}
	\end{center}
	%\end{figure}
\end{wrapfigure}

Из приведенного выражения можно сделать выводы, что для маленьких значений аргумента  функция интегральной экспоненты $E_1(x)$ может быть аппроксимирована логарифмической зависимостью. 

$$E_1(x) = -ln(x) - \gamma $$

График сравнения функций $E_1(x)$ и $ln(x)$ показан на рисунке \ref{ris:ei2}. Видно, что хорошей аппроксимация будет только для маленьких значений аргумента $x < 0.01$. Но для решения уравнения фильтрации именно эта зона представляет наибольший интерес.

%\begin{wrapfigure}{r}{0.5\textwidth}
\begin{figure}[h!]
	\begin{center}
		\begin{tikzpicture}
			\begin{axis}
				[axis lines = left,
				width = 0.98\textwidth,
				xlabel=$x$,
				ylabel={$f(x)$},
				xmode=log,
				log ticks with fixed point,
				]
				\addplot gnuplot[no markers, samples=100, domain = 0.00001:20 ]{expint(1,x)};
				\addlegendentry{$Ei_1(x)$}
				\addplot gnuplot[no markers, samples=100, domain = 0.00001:20]{-log(x)-0.5772};
				\addlegendentry{$ln(x)$}
			\end{axis}
		\end{tikzpicture}
		\caption{Сравнение функций интегральной экспоненты $E_1(x)$ и $ln(x)$ в логарифмическом масштабе.Можно оценить диапазон применимости логарифмической аппроксимации.}
		\label{ris:ei3}
	\end{center}
\end{figure}
%\end{wrapfigure}

Представление интегральной экспоненты в виде логарифмической аппроксимации удобно на практике, так как логарифм легче вычислять. В большинстве языков программирования и инструментов для проведения расчетов расчет логарифма реализован по умолчанию. А для расчета интегральной экспоненты, часто приходится предпринимать дополнительные шаги.

Решение уравнения фильтрации для линейного стока с учетом логарифмической аппроксимации можно представить в виде 
$$ p_D(r_D,t_D) = \frac{1}{2} \left( ln \left( \dfrac{ t_D }{r_D^2}  \right) +0.809 \right) $$

при использовании данного уравнения, следует помнить, что приближенное решение применимо при $\dfrac{r_D^2}{4t_D} < 0.01$

Решение линейного стока в размерных переменных
$$ p\left(r,t\right)=p_i-\frac{18.41q_sB\mu}{kh}\left(-\frac{1}{2}Ei\left(-\frac{\varphi\mu c_tr^2}{0.00144kt}\right)\right) $$

Решение с учетом логарифмической аппроксимации в размерных переменных
$$ p\left(r,t\right)=p_i-\frac{9.205q_sB\mu}{kh}\left(ln{\frac{kt}{\varphi\mu c_tr^2}}-7.12\right)$$
верно при 
$$\frac{kt}{\varphi\mu c_tr^2}>70000 $$

Решения приведены для практических метрических единиц измерения, что можно увидеть по размерному коэффициенту. 

Скин фактор и нестационарное решение
$$ P(r, t) = P{t} - \frac {9.205\mu {q_s} B }{k h}(\ ln\frac {k t}{ \phi \mu {c_t} {r^2}} +7.12 + 2S) $$


%\subsubsection{Радиус исследования}
%Надо бы тут описать концепцию радиуса исследований и подходы к его оценке



\subsubsection{Решение для конечного радиуса скважины}
Для получения сложных решений уравнения фильтрации часто используется преобразование Лапласа.

$$ L \left [ f(t) \right] = \tilde{f}(s) = \int_{0}^{\infty}f(t)e^{-st}dt $$

где $s$ параметр пространства Лапласа соответствующий времени

Решение для бесконечно малого радиуса скважины в пространстве Лапласа будет иметь вид

$$ \tilde{p}_D(s) = \frac{1}{s} K_0 \left( r_D \sqrt s  \right) $$

решение для конечного радиуса скважины

$$ \tilde{p}_D(s) = \frac{K_0 \left( r_D \sqrt{s}  \right) }{ s \sqrt{s} K_1 \left( \sqrt s  \right)  } $$

где 

$K_0$, $K_1$ - модифицированные функции Бесселя 

Перевод решения из пространства Лапласа в обычное пространство не всегда возможен аналитически. В современных условиях перевод делается численно с использованием компьютеров, что позволяет строить и исследования решения уравнения фильтрации при различных условиях. 

Широко распространено применения алгоритма Стефеста для численного обратного преобразования Лапласа. 


Тут будут задачи по гидродинамическим исследованиям.

\chapter{ГДИС. Интерпретация}

Интерпретация ГДИС - решение обратной задачи, когда по наблюдениям требуется найти модель и ее параметры описывающие наблюдения.

\section{Метод прямых линий}

\section{Задачи по интерпретации ГДИС}

\subsection{Простое исследование на неустановившемся режиме. Тест на падение давления на добывающей скважине.}

\subsubsection{Условие задачи}
Задана добывающая скважина, работающая чистой нефтью, со следующими параметрами (приведены в таблице \ref{table:test_data1})).

\begin{table}[h!]
	%\ttabbox
	%	\centering
	{
		\caption{Исходные данные для задачи}
		\label{table:test_data1}
	}
	{
		\begin{tabular}{|l|c|c|l|}
			\hline
			\rowcolor{lightgray}	Параметр& Обозначение &Диапазон &Размерность \\
			\hline
			Радиус скважины&  $r_{w} $&$ 0.1 $ &м \\
			\hline
			Эффективная мощность & $h $&$ 10$ &м  \\
			\hline
			Пористость & $\varphi $&$ 0.15 $&  \\
			\hline
			Общая сжимаемость& $c_{t} $&$ 4\cdot 10^{-5}  $ & 1/бар  \\
			\hline
			Вязкость нефти& $\mu $&$ 0.8$& сП  \\
			\hline
			Объемный коэффициент   &$B $&$ 1.2$& м$^3$/м$^3$  \\
			\hline
			Пластовое давление   &$P_{res} $&$ 250$& атм  \\
			\hline
		\end{tabular}
		
	}
\end{table}

Скважина в течении длительного времени работала с дебитом $Q_{oil} = 80$ м$^3$/сут. Затем была остановлена на КВД. После остановки проводился замер динамики изменения забойного давления, результаты замеров приведены на рисунке \ref{ris:test01}. Исходные данные можно скачать в приложении.


\begin{figure}[h]
	\caption{Измеренные данные для задачи 1. График от времени}
	\label{ris:test01}		
	\begin{tikzpicture}
		\begin{axis}[
			width=\textwidth,
			height = 0.5 \textwidth,
			ylabel={давление, бар}]
			\addplot[color=blue, mark=*] table [y=Pressure, x=Time]{data/test01/pressure.txt};
			\legend{замеры давления}
		\end{axis}
	
	\end{tikzpicture}
	\begin{tikzpicture}
		\begin{axis}[
			width=\textwidth,
			height = 0.3 \textwidth,
			xlabel={время, часы},
			ylabel={дебит, м$^3$/сут},
			legend pos=south east]
			\addplot[color=red, mark=*]  table [y=LiquidRate, x=Time]{data/test01/rates.txt};
			\legend{замеры дебита}
		\end{axis}
	\end{tikzpicture}
\end{figure}

% здесь S=6, k = 30 мД, Cs = 0.1 m3/bar

Оцените величины проницаемости пласта $k$, скин-фактора $S$ и коэффициента послепритока $C_s$. Постройте график демонстрирующий совпадение кривых исходных и модельных с подобранными параметрами в обычных, полулогарифмических и двойных логарифмических координатах. На графике в двойных логарифмических координатах также отобразите графики логарифмической производной. Отобразите на графиках характерные линии для радиального притока и послепритока.


\subsubsection{Рекомендация - дополнительное задание}

Для выполнения расчетов удобно создать расчетный модуль в Excel в который можно было бы загрузить исходные данные и быстро получить ответ.

Ответ относительно легко можно получить двумя способами - методом прямых линий (метод МДХ) и с использованием нелинейной регрессии (используйте поиск решения в Excel).


\subsection{Простое исследование на неустановившемся режиме. Тест на восстановление давления на добывающей скважине.}

\subsubsection{Условие задачи}
Задана добывающая скважина, работающая чистой нефтью, со следующими параметрами (приведены в таблице \ref{table:test_data2})).

\begin{table}[h!]
	%\ttabbox
	%	\centering
	{
		\caption{Исходные данные для задачи}
		\label{table:test_data2}
	}
	{
		\begin{tabular}{|l|c|c|l|}
			\hline
			\rowcolor{lightgray}	Параметр& Обозначение &Диапазон &Размерность \\
			\hline
			Радиус скважины&  $r_{w} $&$ 0.1 $ &м \\
			\hline
			Эффективная мощность & $h $&$ 10$ &м  \\
			\hline
			Пористость & $\varphi $&$ 0.15 $&  \\
			\hline
			Общая сжимаемость& $c_{t} $&$ 4\cdot 10^{-5}  $ & 1/бар  \\
			\hline
			Вязкость нефти& $\mu $&$ 0.8$& сП  \\
			\hline
			Объемный коэффициент   &$B $&$ 1.2$& м$^3$/м$^3$  \\
			\hline
		\end{tabular}
		
	}
\end{table}

Скважина работала более 1000 часов со стабильным дебитом, затем была остановлена на 2 суток. За это время были записаны данные по изменению забойного давления (рисунок \ref{ris:test02}). По данным изменения забойного давления во время остановки оцените величины проницаемости пласта $k$, скин-фактора $S$ и коэффициента послепритока $C_s$ и пластового давления $P_{res} $. Постройте график демонстрирующий совпадение кривых исходных и модельных с подобранными параметрами в обычных, полулогарифмических и двойных логарифмических координатах. На графике в двойных логарифмических координатах также отобразите графики логарифмической производной. Отобразите на графиках характерные линии для радиального притока и послепритока.

\begin{figure}[h]
	\caption{Измеренные данные для задачи 2. График от времени}
	\label{ris:test02}		
	\begin{tikzpicture}
		\begin{axis}[
			x tick label style={/pgf/number format/.cd,%
									scaled y ticks = false,
									set thousands separator={},
									fixed},
			width=\textwidth,
			height = 0.5 \textwidth,
			ylabel={давление, бар}]
			\addplot[color=blue, mark=*] table [y=Pressure, x=Time]{data/test02/pressure.txt};
			\addlegendentry{замеры давления}
		\end{axis}
		
	\end{tikzpicture}
	\begin{tikzpicture}
		\begin{axis}[
			x tick label style={/pgf/number format/.cd,%
				scaled y ticks = false,
				set thousands separator={},
				fixed},
			width=\textwidth,
			height = 0.3 \textwidth,
			xlabel={время, часы},
			ylabel={дебит, м$^3$/сут},
			legend pos=south east]
			\addplot[color=red, mark=*]  table [y=LiquidRate, x=Time]{data/test02/rates.txt};
			\addlegendentry{замеры дебита}
		\end{axis}
	\end{tikzpicture}
\end{figure}

\chapter{Гидропрослушивание. Прокси модели}

\section{Анализ взаимного влияния скважин. Гидропрослушивание}

Здесь будет описание про возможность анализа взаимного влияния скважин при работе. 
Описание простой модели гидропрослушивания и то, как с этой моделью можно работать.

\section{Задачи на гидропрослушивание}

Простые задачи 

1. Построить кривые изменения давления на расстоянии. Сравнить с гидродинамическим симулятором. Построить эталонные кривые

2. Построить кривые изменения давления для нескольких скважин

3. Провести интерпретацию изменения давления в наблюдательной скважине

\section{Прокси модели}

Для анализа взаимного влияния скважин используются гидродинамические модели.

Особый класс подобных моделей получил название прокси моделей

примеры - CRM, модель трубок тока, метод источников и так далее


\section{Задачи на прокси модели}

Построить простую CRM модель на основе данных гидродинамического моделирования



%\insertbibliofull
\printbibliography

\end{document}
