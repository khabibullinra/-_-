\setcounter{secnumdepth}{2}

\usepackage{epigraph}
\usepackage{polyglossia}

\setmainlanguage[babelshorthands=true]{russian}    % Язык по-умолчанию русский с поддержкой приятных команд пакета babel
\setotherlanguage{english}                         % Дополнительный язык = английский (в американской вариации по-умолчанию)

%\setdefaultlanguage{russian}
%\setmainfont{CMU Serif}

\setmonofont{Courier New}                          % моноширинный шрифт
\newfontfamily\cyrillicfonttt{Courier New}         % моноширинный шрифт для кириллицы
\defaultfontfeatures{Ligatures=TeX}                % стандартные лигатуры TeX, замены нескольких дефисов на тире и т. п. Настройки моноширинного шрифта должны идти до этой строки, чтобы при врезках кода программ в коде не применялись лигатуры и замены дефисов
\setmainfont{Times New Roman}                      % Шрифт с засечками
\newfontfamily\cyrillicfont{Times New Roman}       % Шрифт с засечками для кириллицы
\setsansfont{Arial}                                % Шрифт без засечек
\newfontfamily\cyrillicfontsf{Arial}               % Шрифт без засечек для кириллицы

%\setmonofont{LiberationMono}[Scale=0.87]            % моноширинный шрифт
%\newfontfamily\cyrillicfonttt{LiberationMono}[Scale=0.87]   % моноширинный шрифт для кириллицы
    
%\defaultfontfeatures{Ligatures=TeX}                % стандартные лигатуры TeX, замены нескольких дефисов на тире и т. п. Настройки моноширинного шрифта должны идти до этой строки, чтобы при врезках кода программ в коде не применялись лигатуры и замены дефисов
%\setmainfont{LiberationSerif}                      % Шрифт с засечками
%\setmainfont{CMU Serif}
%\newfontfamily\cyrillicfont{LiberationSerif}       % Шрифт с засечками для кириллицы
%\setsansfont{LiberationSans}                       % Шрифт без засечек
%\newfontfamily\cyrillicfontsf{LiberationSans}      % Шрифт без засечек для кириллицы


\usepackage{amsthm}
\usepackage{url} 
\usepackage{hyperref} 


% графика
\usepackage{graphicx}
\usepackage{tikz}                   % Продвинутый пакет векторной графики
\usepackage{pgfplots}
\pgfplotsset{compat=newest}

\usepackage{wrapfig}

\usepackage{physics}
\usepackage{amsmath}
\usepackage{tikz}
\usepackage{mathdots}
\usepackage{yhmath}
\usepackage{cancel}
\usepackage{color}
\usepackage{siunitx}
\usepackage{array}
\usepackage{multirow}
\usepackage{amssymb}
\usepackage{gensymb}
\usepackage{tabularx}
\usepackage{booktabs}
\usetikzlibrary{fadings}
\usetikzlibrary{patterns}
\usetikzlibrary{shadows.blur}
\usetikzlibrary{shapes}


